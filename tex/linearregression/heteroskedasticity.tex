% Copyright © 2013, Gray Calhoun.  Permission is granted to copy,
% distribute and/or modify this document under the terms of the GNU
% Free Documentation License, Version 1.3 or any later version
% published by the Free Software Foundation; with no Invariant
% Sections, no Front-Cover Texts, and no Back-Cover Texts.  A copy of
% the license is included in the section entitled "GNU Free
% Documentation License."

\part*{Heteroskedasticity}%
\addcontentsline{toc}{part}{Heteroskedasticity}

\section{Robust standard errors}

\begin{itemize}
\item Reading for this section is \citet[8.1, 8.2, 8.4]{Gre_2011}.
\end{itemize}

\subsection{introduction}

\begin{itemize}
\item what happens if $E(\epsilon \epsilon' \mid X) \neq \sigma^2 I$?
\item maybe instead we have
\begin{description}
\item[zero correlation] $E(\epsilon_i \epsilon_j \mid X) = 0$
\item[heteroskedasticity] $E(\epsilon_i^2 \mid X) =
            \sigma_i^2(x_i)$ where the $\sigma_i^2$ are all
            (uniformly) finite and positive.
\end{description}
\item we want to know which of our previous conclusions hold
\end{itemize}

\paragraph{examples}
\begin{itemize}
\item may have (say) quarterly earnings data on different firms, and
        you might expect that earnings in certain industries are more
        volatile than others
\item may have GDP data; we'd expect that percent-change in GDP is
        going to be more volatile some quarters compared to others
\end{itemize}

\paragraph{unbiasedness}
\begin{itemize}
\item For unbiasedness, we just used the fact that $E(\epsilon \mid
        X) = 0$
\begin{itemize}
\item exogeneity of the regressors
\end{itemize}
\item unbiasedness holds with heteroskedasticy errors
\end{itemize}

\paragraph{consistency}
\begin{itemize}
\item To prove consistency, we just needed to show that
        $n^{-1}\sum_{i=1}^n x_i \epsilon_i$ has mean zero and obeys a
        law of large numbers
\item zero mean holds just like it does for unbiasedness
\item for LLN, we can show that each $x_i \epsilon_i$ has finite
        variance;
\item observe that
        \[var(x_i \epsilon_i) = E var(x_i \epsilon_i \mid X) = E(x_i
        x_i') \sigma_i^2\]
        which is finite
\item so $\hat\beta$ is consistent even under heteroskedasticity
\end{itemize}

\paragraph{formula for the variance}
\begin{itemize}
\item our original formula for the variance of $\hat\beta$ is
        $\sigma^2 (X'X)^{-1}$
\begin{itemize}
\item obviously, this is a problem since we don't have a constant
          value of $\sigma^2$ any more.
\end{itemize}
\item we can calculate the variance of $\hat\beta$ under our new
        assumption:
\begin{itemize}
\item $E((\hat\beta - \beta)(\hat\beta - \beta)' \mid X) =
          (X'X)^{-1} E(X'\epsilon \epsilon'X \mid X) (X'X)^{-1}$
\begin{itemize}
\item $= (X'X)^{-1} X'DX (X'X)^{-1}$
\begin{itemize}
\item $D$ denotes the diagonal matrix with elements $\sigma_i^2$
\end{itemize}
\item $= (X'X)^{-1} (\sum_{i=1}^n x_i x_i' \sigma_i^2)(X'X)^{-1}$
\end{itemize}
\end{itemize}
\item This matrix is different than $\sigma^2(X'X)^{-1}$
\item any hypothesis tests that we conduct assuming homoskedasticity
        will be invalid under heteroskedasticity
\end{itemize}

\paragraph{finite sample normality}
\begin{itemize}
\item Suppose now that $\epsilon_i \sim N(0, \sigma_i^2)$ given X
\item We know that $\hat\beta - \beta = (X'X)^{-1}X'\epsilon$ which
        is still just a linear transformation of normal random variables
\item So $\hat\beta$ is still normal (given X)
\begin{itemize}
\item variance-covariance matrix is obviously different than it was
          earlier
\item $\hat \beta \sim N\left(0, (X'X)^{-1} \left(\sum_{i=1}^n x_i x_i' \sigma_i^2 \right)
          (X'X)^{-1}\right)$ given $X$
\end{itemize}
\end{itemize}

\paragraph{asymptotic normality}
\begin{itemize}
\item To prove asymptotic normality, we needed to show that
\begin{itemize}
\item $n^{-1} X'X$ obeyed a law of large numbers
\begin{itemize}
\item we assumed this
\item doesn't have anything to do with the errors
\end{itemize}
\item $n^{-1/2} \sum_{i=1}^n x_i \epsilon_i$ obeys a central limit
          theorem
\end{itemize}
\item If you look back at the proof, $x_i \epsilon_i$ has
        heteroskedasticity even if $\epsilon_i$ has constant variances
\item The exact same CLT we used earlier applies here
\begin{itemize}
\item we used the Lindeberg-Feller CLT earlier
\end{itemize}
\item So we have asymptotic normality under heteroskedasticity
\begin{itemize}
\item let $Q = \lim n^{-1} X'X$
\item let $\Omega = \lim n^{-1} \sum_i x_i x_i' \sigma_i^2$
\item $\sqrt{n}(\hat\beta - \beta) \to N(0, Q^{-1} \Omega Q^{-1})$ in
          distribution.
\end{itemize}
\end{itemize}

\paragraph{summary}
\begin{itemize}
\item heteroskedasticity only affects the variance of $\hat\beta$
\item it will make the tests we've studied invalid
\item if we can estimate the new variance
\begin{itemize}
\item we can replace the old variance in formulae for test statistics
\item these new tests will be (asymptotically) valid
\end{itemize}
\end{itemize}

\subsection{White's heteroscedasticity robust estimator}
\begin{itemize}
\item The goal is going to be getting new test statistics
\item To do that, we're going to work out an estimator for the
       asymptotic variance-covariance matrix, $Q^{-1} \Omega Q^{-1}$
\item Derived by \citet{Whi_1980}
\item We can replace $s^2(X'X)^{-1}$ with this new estimator to get
       aysmptotically valid versions of the t-test and the F-test
\item reading is 4.3 of \citet{KlZ_2008}.
\end{itemize}

\paragraph{split up the sandwich}
\begin{itemize}
\item goals
\begin{itemize}
\item We need to estimate $Q^{-1}$
\item We need to estimate $\Omega$.
\end{itemize}
\item $Q^{-1}$ is easy: we know that $n^{-1} X'X$ is a consistent
        estimator of $Q$
\item For $\Omega$, we're going to apply our usual trick, the LLN.
\begin{itemize}
\item $\Omega_n = n^{-1} \sum E(\epsilon_i^2 \mid x_i) x_i x_i'$
\item $\hat\Omega = \lim_n n^{-1} \sum \hat\epsilon_i^2 x_i x_i'$
\end{itemize}
\item want to prove that $\hat\Omega - \Omega_n \to 0$ in probability
\end{itemize}

\paragraph{consistency of $\hat\Omega$}
\begin{itemize}
\item we can rewrite the difference:
        \[\hat\Omega - \Omega_n = n^{-1} \sum_i x_i x_i'
        (\hat\epsilon_i^2 - \sigma_i^2)\]
\item now, do our favorite trick
        \[= n^{-1} \sum_i x_i x_i'(\hat\epsilon_i^2 - \epsilon_i^2) + 
        n^{-1} \sum_i x_i x_i'(\epsilon_i^2 - \sigma_i^2)\]
\end{itemize}

\paragraph{LLN for the second term}
\begin{itemize}
\item We know that each summand has mean zero
         \[E(x_i x_i'(\epsilon_i^2 - \sigma_i^2))  = E(x_i
         x_i' E(\epsilon_i^2 - \sigma_i^2 \mid x_i)) = 0\]
\item Under conditions similar to those we used earlier, we know
         that $x_i x_i' (\epsilon_i^2 - \sigma_i^2)$ has finite and
         positive variance.
\item so the sum obeys (for example) Chebychev's LLN
\end{itemize}

\paragraph{convergence of the first term}
\begin{itemize}
\item note that
         \[ \hat\epsilon_i^2 = (y_i - x_i'\hat\beta)^2 = (\epsilon_i +
         x_i'\beta - x_i'\hat\beta)^2 = \epsilon_i^2 - 2 \epsilon_i
         x_i'(\hat\beta - \beta) + (x_i'(\hat\beta - \beta))^2\]
\item so the first average becomes
         \[n^{-1}\sum x_i x_i' (x_i'(\hat\beta - \beta))^2 - 2 n^{-1}
         \sum  x_i x_i' (\epsilon_i x_i'(\hat\beta - \beta)) \]
\item (informally) we can see why these terms converge to zero
\begin{itemize}
\item look the (j,k) element of the first matrix
           \[ n^{-1} \sum_i x_{ij} x_{ik}(x_i'(\hat\beta - \beta))^2 =
           (\hat\beta - \beta)' \left(n^{-1} \sum_i x_i x_{ij} x_{ik}
           x_{i}'\right) (\hat\beta- \beta)\]
\begin{itemize}
\item under usual conditions, the term inside the sum is going to
             converge to its average
\item each $\hat \beta - \beta$ is going to converge to zero
\end{itemize}
\item look at the (j,k) element of the second matrix
           \[ n^{-1} \sum_i x_{ij} x_{ik}(\epsilon_i x_i'(\hat\beta - \beta)) =
           \left(n^{-1} \sum_i \epsilon_i x_{ij} x_{ik}
           x_{i}'\right) (\hat\beta- \beta)\]
\begin{itemize}
\item the term inside the sum is going to converge to its
             average (zero, in this case)
\item $\hat\beta - \beta$ is going to converge to zero
\end{itemize}
\end{itemize}
\end{itemize}

\paragraph{conclusion}
\begin{itemize}
\item consequently $\Omega - \hat\Omega = \Omega - \Omega_n +
         \Omega_n - \hat\Omega \to 0$ in probability
\end{itemize}

\subsection{uses of the robust variance-covariance matrix}

\begin{itemize}
\item Define $\hat\Sigma$ as 
       \[\hat\Sigma = (n^{-1} X'X)^{-1} n^{-1} \sum_i x_i x_i' \hat\epsilon_i^2 (n^{-1} X'X)^{-1}\]
\item variation of the t-test
\begin{itemize}
\item $\sqrt{n} \frac{\hat\beta_j - \beta_j}{\hat\Sigma_{jj}} \to
         N(0,1)$ in distribution
\end{itemize}
\item variation of the F-test (called the wald test)
\begin{itemize}
\item suppose that $R$ is a $J \times K$ matrix with full rank
\item $n \cdot (R(\hat\beta - \beta))'(R\hat\Sigma R')^{-1}
         (R(\hat\beta - \beta))$ is asymptotically chi-square with $J$
         degrees of freedom
\end{itemize}
\item obviously, you can use this for a hypothesis test by imposing
       $R\beta  = q$
\end{itemize}

\section{testing for heteroskedasticity}

\begin{itemize}
\item reading is 8.5 of \citet{Gre_2011} and 4.2 of \citet{KlZ_2008}
\item we're not going to work through the proofs of this section, but
      we will explain the intuition behind them
\end{itemize}

\subsection{heuristic idea for test}

\begin{itemize}
\item $\hat\beta$ is consistent even under heteroskedasticity
\item $\hat\epsilon_i \to \epsilon_i$ for each $i$.
\item If $\sigma_i^2$ is different for different individuals, it
       implies that $\epsilon_i^2$ should be correlated with $x_i$.
\item Why not regress $\epsilon_i^2$ on $x_i$ and do an F-test for
       the significance of the overall regression?
\end{itemize}

\subsection{more formal idea for test}

\begin{itemize}
\item Want to test the null hypothesis
       \[ H_o: \quad \sigma_i^2 = \sigma \qquad i = 1,\dots,n \]
\item If the null hypothesis is true, then $s^2 (n^{-1} X'X)^{-1} \to
       \Omega$ in probability as $n\to\infty$
\begin{itemize}
\item $\Omega = \lim n^{-1} \sum_i x_i x_i' \sigma_i^2$.
\item this equals $\sigma^2 \lim n^{-1} \sum_i x_i x_i'$ for
         homoskedasticity
\end{itemize}
\item We know that $n^{-1} \sum_i \hat\epsilon_i^2 x_i x_i' \to
       \Omega$ as well
\begin{itemize}
\item holds with or without heteroskedasticity.
\end{itemize}
\item So, if the null hypothesis is true, $n^{-1} (\sum_i
       \hat\epsilon_i^2 - s^2) x_i x_i' \to \Omega$ in probability.
\begin{itemize}
\item if the null is false, this doesn't happen.
\end{itemize}
\item This looks like an average
\end{itemize}

\subsection{explanation of the test}

\begin{itemize}
\item Let $\psi_i$ denote the vector of unique elements of $x_ix_i'$,
       along with a constant term (if not in $x_i$).
\begin{itemize}
\item say that $\psi_i$ has length $p$.
\end{itemize}
\item We'd expect that $n^{-1/2} \sum_i (\hat\epsilon_i^2 - s^2)
       \psi_i$ should obey a central limit theorem, so $\to N(0, V)$
       (say) in distribution (where $V$ is a pretty large matrix).
\item Then we'd have 
       \[\left(n^{-1/2} \sum_i (\hat\epsilon_i^2 - s^2)\psi_i\right)'
       \hat V^{-1} \left(n^{-1/2} \sum_i (\hat\epsilon_i^2 -
       s^2)\psi_i\right) \xrightarrow{d} \chi_p^2\]
\begin{itemize}
\item holds under the null hypothesis
\item fails under the alternative (which gives us power).
\end{itemize}
\end{itemize}

\subsection{implementation of the test}

     Statistic is quite easy to implement (derived by \citealp{Whi_1980})

\begin{enumerate}
\item Regress $y_i$ on $x_i$ and save the OLS residuals.
\item Regress $\hat\epsilon_i$ on $\phi_i$ and calculate the $R^2$
        from this regression
\item $n R^2$ is asymptotically $chi^2_{p-1}$ under the null
        hypothesis.
\end{enumerate}

\section{Generalized Least Squares}

Reading is \citet[8.3]{Gre_2011}

\subsection{motivation}

\begin{itemize}
\item Suppose that instead of having $E( \epsilon \epsilon \mid X) =
       \sigma^2 I$ we \textbf{knew} that $E(\epsilon \epsilon \mid X) =
       \sigma^2 \Omega$ for some other known matrix $\Omega$.
\item we could still do MLE to estimate $\beta$ and $\sigma$
\item Would generally give you a different estimate of $\beta$ than
       OLS.
\end{itemize}n

\subsection{Infeasible GLS}

\paragraph{Draw pictures}

\paragraph{mathematics}
\begin{itemize}
\item We know that $\Omega^{-1/2}\epsilon \sim (0, \sigma^2 I)$ given $X$
\item What happens if we regress $\Omega^{-1/2} Y$ on $\Omega^{-1/2} X$?
\begin{itemize}
\item Let $\tilde Y = \Omega^{-1/2} Y$
\item Let $\tilde X = \Omega^{-1/2} X$
\item $\tilde\epsilon = \tilde Y - \tilde X \beta$
\end{itemize}
\item If $E(Y \mid X) = X\beta$ then $E(\tilde Y \mid X) = \tilde X
        \beta$
\item If we regress $\tilde Y$ on $\tilde X$, we're back in standard
        OLS with homoskedastic errors.
\item Estimator is $\hat\beta_{GLS} =
        (X'\Omega^{-1}X)^{-1}X'\Omega^{-1}Y$
\begin{itemize}
\item Unbiased (obviously)
\item Satisfies Gauss-Markov (BLUE)
\item Variance is $\sigma^2 (\tilde X' \tilde X)^{-1} = \sigma^2
          (X' \Omega^{-1} X)^{-1}$
\item Normal if the error terms are normal
\item Asymptotically normal otherwise.
\end{itemize}
\end{itemize}

\subsection{Weighted Least Squares (ie example)}

\begin{itemize}
\item Suppose that we believe that $\sigma_i^2 = \sigma^2 \cdot
       x_{ij}^2$ for one of the regressors $j$.
\begin{itemize}
\item textbook gives an example: dependent variable is firm profits
         and indep variable is size.
\end{itemize}
\item To implement GLS, we want to regress
       \[ y/x_k = \beta + \beta_1(x_1/x_k) + \beta_2(x_2/x_k) +
       \cdots + \epsilon/x_k \]
\begin{itemize}
\item intuitively, if the firm is bigger, profit is more volatile
         and so we want to count that observation less in our estimation.
\end{itemize}
\end{itemize}

\subsection{Feasible GLS}

\begin{itemize}
\item Suppose that $\Omega$ has a few unknown parameters:
\begin{itemize}
\item Two examples from TS:
\begin{itemize}
\item $\Omega = (1, \rho, \rho^2, \dots)$ (as
           a matrix)
\item $\Omega = (1, \gamma, 0, 0; \gamma, 1, ...)$ (as a matrix)
\end{itemize}
\end{itemize}
\item We could just estimate those parameters along with the others.
\item Two different approaches
\begin{itemize}
\item MLE (next semester)
\item two-step least squares (FGLS)
\end{itemize}
\end{itemize}

\paragraph{Explanation of approach}
\begin{itemize}
\item Remember how we tested for heteroskedasticity
\item regress $\hat\epsilon$ on the (squared) regressors and test
        for significance
\item use a similar approach here, but use for modeling.
\item Suppose that
        \[ E(\epsilon_i^2 \mid X) = z_i'\alpha \]
        where $z_i$ is some function of the regressors.
\item If we could regress $\epsilon_i^2$ on $z_i$, we could estimate
        $\alpha$ consistently.
\item Since $\hat\epsilon_i$ is consistent for $\epsilon_i$, we can
        regress $\epsilon_i^2$ on $z_i$ to estimate $\alpha$.
\end{itemize}

\paragraph{Algorithm}
\begin{itemize}
\item Regress $y$ on $x$ to get $\hat\epsilon$
\item Regress $\hat \epsilon$ on $z$ to estimate $\alpha$
\item Regress $y_i/w_i$ on $x_i/w_i$ to estimate $\hat\beta_{FGLS}$
        where $w_i = sqrt{z_i'\hat\alpha}$
\end{itemize}

\paragraph{some more math}
\begin{itemize}
\item if we let $\hat\Omega$ be the estimate of $\Omega$ that uses
        $\hat\alpha$, consistency of $\hat\alpha$ ensures that
        $\hat\Omega$ behaves asymptotically like $\Omega$.
\item All of the asymptotic results from the GLS estimator apply to
        the Feasible GLS estimator
\item Finite sample results don't necessarily hold.
\end{itemize}

%%% Local Variables:
%%% mode: latex
%%% TeX-master: "../../linearregression"
%%% End:

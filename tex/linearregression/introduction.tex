% Copyright © 2013, authors of the "Econometrics Core" textbook; a
% complete list of authors is available in the file AUTHORS.tex.

% Permission is granted to copy, distribute and/or modify this
% document under the terms of the GNU Free Documentation License,
% Version 1.3 or any later version published by the Free Software
% Foundation; with no Invariant Sections, no Front-Cover Texts, and no
% Back-Cover Texts.  A copy of the license is included in the file
% LICENSE.tex and is also available online at
% <http://www.gnu.org/copyleft/fdl.html>.

\part*{Introduction}%
\addcontentsline{toc}{part}{Introduction}

\paragraph{description of the goals}
\begin{itemize}
\item What are we hoping to do?
\begin{itemize}
\item we probably have in mind changing one of these variables
\begin{itemize}
\item conceivably, could change number of teachers per student
\item conceivably, could change money available per student
\end{itemize}
\item so, we want something like (informally) E(change in reading
          score | an independent body makes a 1\% change in expenditure)
\end{itemize}
\item what makes econometrics distinctive?
\begin{itemize}
\item we \underline{don't} always get to observe changes in expenditure and
          changes in scores
\begin{itemize}
\item sometimes we do
\item sometimes we see levels of expenditure and levels scores
\end{itemize}
\item we almost never get to observe \underline{changes made by an independent           body}.
\begin{itemize}
\item often, you'll see changes that were made for some reason
\end{itemize}
\end{itemize}
\end{itemize}

\paragraph{\underline{the key strategy for empirical research}}
\begin{itemize}
\item an important message before we start
\begin{itemize}
\item think about what experiment you would \textbf{want to design} if you
          could
\begin{itemize}
\item one approach if we want to understand how expenditure affects student
            reading test scores, we'd want to take all of the schools
\item \textbf{randomly} and \textbf{independently} change their funding (leave
            some unchanged)
\item observe the change in test scores the next year
\end{itemize}
\item then think hard about how closely the process that created
          your data matches that experiment
\begin{itemize}
\item this can be an iterative process
\end{itemize}
\item if you can not think of an experiment, the question might not
          be an empirical question
\begin{itemize}
\item sometimes this happens when you think about changing
            preference parameters
\end{itemize}
\end{itemize}
\end{itemize}

\paragraph{description of the ``environment''}
\begin{itemize}
\item have n data points (observations)
\item have k+1 regressors
\item believe that there's a relationship between the dependent
        variable and the regressors:
        \[\begin{pmatrix} Y_1 \\ \vdots \\ Y_n\end{pmatrix}
        = \begin{pmatrix} 
        X_{1,0} & X_{1,1} & \cdots & X_{1,k} \\
        \vdots \\
        X_{n,0} & X_{n,1} & \cdots & X_{n,k} \end{pmatrix}
        \begin{pmatrix}  \beta_0 \\ \vdots \\ \beta_k \end{pmatrix}+ 
        \begin{pmatrix} \varepsilon_1 \\ \vdots \\
          \varepsilon_n \end{pmatrix} \]
\begin{itemize}
\item $Y$ is the response vector
\item $X$ is the design matrix
\item $\beta$ is the vector of unknown coefficients
\item $\varepsilon$ is the error
\item so, the ith row of this relationship is
            \[Y_i = X_{i,0} \beta_0 + X_{i,1} \beta_1 + \cdots + X_{i,k}
            \beta_k + \varepsilon_i\]
\end{itemize}
\end{itemize}

\subsection{informal motivation}
\begin{itemize}
\item In matrix form: $Y = X \beta + \varepsilon$
\item if the error is small, then $Y \approx X\beta$
\begin{itemize}
\item so $X'Y \approx X'X \beta$
\begin{itemize}
\item $X'X$ will be invertible if $X$ has full rank
\end{itemize}
\item and then $\beta \approx (X'X)^{-1} X'Y$
\end{itemize}
\end{itemize}

\subsection{outline of the second half of the course}

\begin{itemize}
\item we're going to talk mostly about this estimator and slight
       variations of it
\begin{itemize}
\item what it does (under what assumptions)
\item what it \textbf{doesn't} do
\end{itemize}
\item next semester, you'll discuss improvements
\end{itemize}

%%% Local Variables: 
%%% mode: latex
%%% TeX-master: "../../linearregression"
%%% End: 

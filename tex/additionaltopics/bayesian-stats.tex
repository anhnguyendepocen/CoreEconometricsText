% Copyright © 2013, Gray Calhoun.  Permission is granted to copy,
% distribute and/or modify this document under the terms of the GNU
% Free Documentation License, Version 1.3 or any later version
% published by the Free Software Foundation; with no Invariant
% Sections, no Front-Cover Texts, and no Back-Cover Texts.  A copy of
% the license is included in the section entitled "GNU Free
% Documentation License."

\part*{Bayesian statistics}%
\addcontentsline{toc}{part}{Bayesian statistics}

\begin{itemize}
\item Covered on
\begin{itemize}
\item \textit{2010-10-12 Tue}, \textit{2010-10-14 Thu}, \textit{2010-10-19 Tue}
\end{itemize}
\item Reading
\begin{itemize}
\item CB 5.6, 7.2.3
\item Greene 16
\end{itemize}
\end{itemize}
\section{introduction to bayesian statistics}
\label{sec-1}

\begin{itemize}
\item For classical statistics (what we've been doing so far), unknown parameters are treated as fixed.
\begin{itemize}
\item reasonable perspective
\item can be limiting
\begin{itemize}
\item What would you do if you have to make a forecast and have no data?
\end{itemize}
\end{itemize}
\item For Bayesian statistics, we treat the unknown parameter as a
      random variable
\end{itemize}
\subsection{mathematics}
\label{sec-1-1}

\begin{itemize}
\item dead simple.
\item denote the whole sample as $X$ (ie $X_1,\dots,X_n$)
\begin{itemize}
\item we'll assume that this is iid, but that's not important
\end{itemize}
\item suppose that $X_i \sim f_X(\cdot; \theta)$
\begin{itemize}
\item this specification is part of the model
\item this part looks like what we've done
\end{itemize}
\item now, suppose that $\theta \sim f_\theta$
\begin{itemize}
\item so the previous line really should be
         \[X_i \mid \theta \sim f_X(\cdot, \theta)\]
\item this density represents what we believe about the unknown
         parameters $\theta$ before we look at any data
\item called the \emph{prior} distribution
\item parameters on this density are called ``hyperparameters''
\end{itemize}
\item We're going to base decisions on the posterior distribution of
       $\theta$, which is updated to reflect what we've learned from
       the data
\begin{itemize}
\item posterior distribution is the conditional distribution
         $f_\theta(t \mid X)$
\item we get this using Bayes's rule:
         \[f_\theta(t \mid X = x) = \frac{f_X(x \mid \theta = t) f_\theta(t)}{f_X(x)}\]
\end{itemize}
\item If we need to represent the posterior by a single number, we
       can do this as long as we have a loss function
\begin{itemize}
\item Remember, a loss function is some function $L$
\begin{itemize}
\item convex
\item $L(0) = 0$
\item these conditions can be generalized
\end{itemize}
\item estimator is the value $\hat\theta$ that minimizes
         \[ E(L(\theta - \hat \theta \mid X_1,\dots,X_n) \]
\begin{itemize}
\item expectation is over $\theta$
\end{itemize}
\item similar to certainty-equivalence.
\end{itemize}
\end{itemize}
\subsection{example}
\label{sec-1-2}

\begin{itemize}
\item Suppose $X_i \sim i.i.d.\ bernoulli(\theta)$
\begin{itemize}
\item we're interested in $\sum_i X_i \sim\ bernoulli(n, \theta)$
\item $f(x \mid \theta) = \binom{n}{x} \theta^x (1-\theta)^{n-x}$
\end{itemize}
\item A prior might be that $\theta \sim\ uniform(0,1)$
\end{itemize}
\section{Choice of prior}
\label{sec-2}

\begin{itemize}
\item There are three ``standard'' methods for choosing a priod
\begin{itemize}
\item Uninformative prior
\begin{itemize}
\item ad hoc
\item formalized
\end{itemize}
\item Conjugate prior
\item carefully considered prior (this is less common)
\end{itemize}
\end{itemize}
\subsection{Uninformative prior}
\label{sec-2-1}

\begin{itemize}
\item One chosen to (in a way) minimize the knowledge embodied in
       the prior
\item Often people use ad hoc uninformative priors
\begin{itemize}
\item can give misleading results
\end{itemize}
\item Formal method of getting uninformative prior:
\begin{itemize}
\item prior density should be proportional to $\sqrt{|I(\theta)|}$
\begin{itemize}
\item ``abs'' is determinant
\item $I(\theta)$ is the `information matrix' from the
           Cramer-Rao lower bound
           \[I_{ij}(\theta) = E[ \frac\partial{\partial \theta_i} \log
           f_X(X \mid \theta) \cdot  \frac\partial{\partial \theta_j} \log
           f_X(X \mid \theta) ]\]
\end{itemize}
\end{itemize}
\item Reference prior
\begin{itemize}
\item Chooses the prior distribution that maximizes the expected
         distance between prior ( $f_\theta(\cdot)$ ) and posterior (
         $f_\theta(\cdot \mid X)$ ) distributions
\item often gives same result as Jeffreys Prior
\end{itemize}
\end{itemize}
\subsection{Conjugate prior}
\label{sec-2-2}

\begin{itemize}
\item A conjugate prior is one that is chosen so that the prior and
       the posterior are in the same family
\begin{itemize}
\item computationally convenient
\end{itemize}
\item Definition: (from Cassella Berger 7.2.3)  Let $\mathcal{F}$ denote the class of pdfs $f_X(x \mid \theta)$.  A class $\Pi$
       of prior distributions is a \emph{conjugate family} for $\mathcal F$ if the posterior is in the class $\Pi$ for all $f$ in $\mathcal F$,
       all priors in $\Pi$, and all $x$.
\item Uniform is a special case of beta distribution with
       parameters (1,1)
\item consider beta prior with parameters $\alpha$ and $\beta$ and
       $X \sim \operatorname{binomial}(n,p)$ DGP
\begin{itemize}
\item $p_\theta(t) = \frac{\Gamma(\alpha + \beta) t^{\alpha-1}
         (1-t)^{\beta-1}}{\Gamma(\alpha) \Gamma(\beta)}$
     \item $p_{X}(S \mid \theta) = \binom{n}{S} \theta^S
       (1-\theta)^{n-S} = \frac{\Gamma(n + 1)}{\Gamma(S + \Gamma(n - S
         + 1)} \theta^S (1-\theta)^{n-s}$
\item To get posterior, combine the distributions as before
\begin{itemize}
\item $p_\theta(t \mid data) 
           = K \times \frac{\Gamma(n + 1)}{\Gamma(S + 1) \Gamma(n - S + 1)} t^S (1-t)^{n-S} \times \frac{\Gamma(\alpha + \beta) t^{\alpha-1} (1-t)^{\beta-1}}{\Gamma(\alpha) \Gamma(\beta)}$
\begin{itemize}
\item $= K \times \frac{\Gamma(n + 1) \Gamma(\alpha + \beta)}{\Gamma(S + 1) \Gamma(n - S + 1) \Gamma(\alpha) \Gamma(\beta)} t^{\alpha + S - 1} (1 - t)^{\beta + n - S - 1}$
\end{itemize}
\item we can recognize that this is the beta($\alpha + S, \beta + n - S$) density again; so we must have 
           \[ K = (\frac{\Gamma(n + 1) \Gamma(\alpha + \beta)}{\Gamma(S + 1) \Gamma(n - S + 1)\Gamma(\alpha)\Gamma(\beta)})^{-1} \frac{\Gamma(\alpha + \beta + n)}{\Gamma(\alpha + S)\Gamma(\beta + n)} \]
\end{itemize}
\item This is the beta distribution again:
\begin{itemize}
\item mean is $(S + \alpha) / (n + \alpha + \beta)$
\end{itemize}
\end{itemize}
\item for any bernoulli, if the prior is $beta(\alpha,\beta)$ the
       posterior is $beta(S + \alpha, N - S + \beta)$
\end{itemize}
\paragraph{advantages}
\label{sec-2-2-1}

\begin{itemize}
\item mathematical convenience
\item interpretation as an augmented dataset
\begin{itemize}
\item Another way to interpret the prior in this case
\begin{itemize}
\item suppose we'd observed other data previously
\item $\alpha$ is the number of successes we observed
\item $\beta$ is the number of failures
\end{itemize}
\item Increasing $\alpha$ and $\beta$ decreases the variance of the prior
\begin{itemize}
\item makes the prior have a larger effect on our estimator
\end{itemize}
\item we see this for some families of distributions (exponential family)
\end{itemize}
\end{itemize}
\subsection{Asymptotics}
\label{sec-2-3}

\begin{itemize}
\item What happens as $N \to \infty$ (for conjugate prior)
\item $\alpha$ and $\beta$ are fixed, so $\frac{R + \alpha}{R + G +
       \alpha + \beta}$ behaves like $\frac{R}{R+G}$
\item ie the prior doesn't matter and the Bayesian estimator
       (squared error loss) and MLE behave the same
\item typically the case for general priors, but one can construct
       counterexamples
\end{itemize}
\section{Implementing the prior numerically}
\label{sec-3}

\begin{itemize}
\item for conjugate priors, everything is easy (if you know the analytical form)
\item otherwise, finding the density is kind of a bitch:
      \[f_\theta(t \mid X = x) \propto f_X(x \mid \theta = t) f_\theta(t)\]
\item On the computer, though, it's easy using metropolis-hastings algorithm
\begin{itemize}
\item remember the accept-reject algorithm we used for homework
\begin{itemize}
\item want to draw from some $X \sim f$ w/ support between 0 and 1,
         height between zero and one.
\begin{itemize}
\item draw candidate $X$ from uniform distribution
\item draw another variable $V$ from uniform distribution
\item if $V \leq f(X)$, keep $X$, otherwise draw another $X$ and $V$.
\end{itemize}
\item can be extended to non-uniform for both
\item problem, though:
\begin{itemize}
\item requires candidate density to be ``larger'' than real density
\begin{description}
\item[→] really talking about in the tails.
\item[→] draw
\end{description}
\end{itemize}
\item metropolis algorithm gets around this restriction
\begin{itemize}
\item one problem: gives asymptotic/approximate results
\end{itemize}
\end{itemize}
\end{itemize}
\end{itemize}
\subsection{description of algorithm (Casella Berger, p. 254)}
\label{sec-3-1}

     Suppose you can generate $V \sim f_V$, and the density $f_Y$
     has the same support
\begin{enumerate}
\item Generate $Z_0 \sim f_V$
\item To generate $Z_i$, $i > 0$:
\begin{itemize}
\item Generate $V_i \sim f_V$
\item Set $Z_i$
\begin{itemize}
\item $= V_i$ with probability $\min\{ f_Y(V_i)/ f_V(V_i)
            \times f_V(Z_{i-1}) / f_Y(Z_{i-1}), 1\}$
\end{itemize}
\item $= Z_{i-1}$ otherwise
\end{itemize}
\item As $i \to \infty$, $Z_i \to f_Y$ in distribution.
\end{enumerate}
\subsection{simple example}
\label{sec-3-2}
\paragraph{ergodicity}
\label{sec-3-2-1}

\begin{itemize}
\item let $V =$
\begin{itemize}
\item 0 with probability 1/4
\item 1 with probability 3/4
\end{itemize}
\item let $Y =$
\begin{itemize}
\item 0 with probability 1/2
\item 1 with probability 1/2
\end{itemize}
\item suppose $Z_{i-1} = 0$ and draw $V_i$
\begin{itemize}
\item if $V_i = 0$, $Z_i = 0$ always
\item if $V_i = 1$, $Z_i$ equals
\begin{itemize}
\item 1 with probability 1/3
\item 0 with probability 2/3
\end{itemize}
\item so $Z_i \mid Z_{i-1} = 0$ is
\begin{itemize}
\item 0 with prob $1/4 + 2/3 \times 3/4 = 3/4$
\item 1 with prob $1/3 \times 3/4 = 1/4$
\end{itemize}
\end{itemize}
\item suppose $Z_{i-1} = 1$ and draw $V_i$
\begin{itemize}
\item if $V_i = 0$, $Z_i$ is
\begin{itemize}
\item 0 with probability $\min(.5/.25 \times .75/.5, 1) =
            \min(3,1) = 1$
\item 1 with prob 0
\end{itemize}
\item if $V_i = 1$, $Z_i$ is 1 always
\item so $Z_i \mid Z_{i-1} = 1$ is
\begin{itemize}
\item 0 with prob 1/4
\item 1 with prob 3/4
\end{itemize}
\end{itemize}
\item Now, suppose we have $Z_{i-1} \sim f_Y$
\begin{itemize}
\item 0 with prob 1/2
\item 1 with prob 1/2
\end{itemize}
\item Then $Pr[Z_i = 0] = P[Z_i = 0 \mid Z_{i-1}=0] P[Z_{i-1} = 0] + P[Z_i = 0 \mid Z_{i-1}=1] P[Z_{i-1} = 1]$
\begin{itemize}
\item $= \frac{3}{4} \frac{1}{2} + \frac{1}{4} \frac{1}{2} = \frac12$
\end{itemize}
\item So once we have a draw from $f_Y$, we can use the
        metropolis-hastings algorithm to generate another draw from $f_y$
\end{itemize}
\paragraph{getting the initial draw from $f_Y$}
\label{sec-3-2-2}

\begin{itemize}
\item we can represent the ``transitions'' as a matrix
\item set $$P = \begin{pmatrix} 3/4 & 1/4 \\ 1/4 & 3/4 \end{pmatrix}$$
\item each row represents $P[Z_i = z \mid Z_{i-1} = z']$
\begin{itemize}
\item different rows give different $z'$
\end{itemize}
\item we can let the row-vector $Z_0 = (1-p, p)$ represent the
        probabilities that the initial $Z_0$ is 0 or 1.
\item Then, $Z_0 P$ gives probabilities that $Z_1$ is zero or 1
\begin{itemize}
\item $Z_0 P^2$ gives probs that $Z_2$ is zero or 1
\item dot dot dot
\item $Z_0 P^i$ gives probs that $Z_i$ is zero or 1
\end{itemize}
\item we can diagonalize $P$
        \[P = Q D Q^{-1}\]
        with
        \[Q = \begin{pmatrix} 1 & -1/4 \\ 1 &
        1/4 \end{pmatrix},\qquad 
        Q^{-1} = \begin{pmatrix} 1/2 & 1/2 \\ -2 & 2 \end{pmatrix},
        \qquad 
        D = \begin{pmatrix} 1 & 0 \\ 0 & 1/2 \end{pmatrix}\]
\begin{itemize}
\item Q gives the right-eigenvectors of P
\item $Q^{-1}$ gives the left eigenvectors of $P$
\item D gives the eigenvalues of P
\end{itemize}
\item note that 
        \[P^{i} = (Q D Q^{-1})^n = Q D^n Q^{-1} = Q \begin{pmatrix}
        1^n & 0 \\ 0 & 0.5^n \end{pmatrix} Q^{-1} \to Q \begin{pmatrix}
        1^n & 0 \\ 0 & 0 \end{pmatrix} = \begin{pmatrix} 0.5 & 0.5
        \\ 0.5 & 0.5 \end{pmatrix}\]
\item so $Z_0 P^n = (1-p, p) P^n \to ((1-p)/2 + p/2, (1-p)/2 +
        p/2) = (1/2, 1/2)$
\begin{itemize}
\item no matter what density we start with, after a large number
          of steps we'll get draws from close to the marginal
          distribution.
\item quick numeric example (going to be much better behaved than
          typical mcmc application)
\begin{itemize}
\item \texttt{mpower <- function(M, power) \{}
\item \texttt{Mp <- M}
\item \texttt{for (i in seq\_len(power-1)) Mp <- Mp \%*\% M}
\item \texttt{Mp\}}
\item \texttt{P <- matrix(c(3/4,1/4,1/4,3/4), 2, 2)}
\item \texttt{c(1,0) \%*\% P}
\item \texttt{c(1,0) \%*\% mpower(P, 3)}
\item \texttt{c(1,0) \%*\% mpower(P, 10)}
\end{itemize}
\end{itemize}
\end{itemize}
\paragraph{wrap up of example}
\label{sec-3-2-3}

\begin{itemize}
\item we made some simplifications
\begin{itemize}
\item finite number of possible values
\end{itemize}
\item basic result holds much more generally (ie continuous rv)
\end{itemize}
\subsection{numeric example}
\label{sec-3-3}

\begin{itemize}
\item data: $X \mid \theta \sim$ binom($10, \theta$)
\begin{itemize}
\item \texttt{n <- 10}
\item \texttt{x <- rbinom(1, n, runif(1))}
\item \texttt{x}
\end{itemize}
\item want to find
\begin{enumerate}
\item posterior distribution
\item minimum risk estimtor for squared-error (ie the mean of the posterior)
\item estimator that minimizes $E(L(p - \hat p) \mid X$ where
          $L(e) = \exp(-e) - 1$ if $e < 0$, \emph{e} if $e \geq 0$
\begin{itemize}
\item \texttt{L2 <- function(e) ifelse(e < 0, exp(-e) - 1, e)}
\item \texttt{curve(L2(x), from = -1, to = 1)}
\end{itemize}
\end{enumerate}
\item prior: $\theta \sim k \phi(x)$ if $x \in (1/4, 3/4)$, 0 otherwise
\begin{itemize}
\item $k$ chosen so that it integrates to 1
\item not necessarily realistic
\item note that Bayes rule tells us that anywhere the prior is zero,
         the posterior will be zero as well, so the posterior's going to be 
         between 1/4 and 3/4
\item need a function that will generate from prior:
\begin{itemize}
\item \texttt{rprior <- function() \{}
\begin{itemize}
\item \texttt{repeat \{}
\item \texttt{x <- rnorm(1)}
\item \texttt{if (x > 0.25 \& x < 0.75) break}
\item \texttt{\}}
\item \texttt{x}
\item \texttt{\}}
\end{itemize}
\item \texttt{hist(replicate(600, rprior()), 40, freq = FALSE, xlim = c(0,1))}
\end{itemize}
\item need a function that will evaluate (proportional to) prior:
\begin{itemize}
\item \texttt{dprior <- function(p) ifelse(p > 0.25 \& p < 0.75, dnorm(p), 0)}
\item \texttt{curve(dprior(x), from = 0, to = 1, n = 1000, add = TRUE)}
\end{itemize}
\item need to evaluate the density of the posterior:
\begin{itemize}
\item \texttt{dposterior <- function(p, x, n) dbinom(x, n, p) * dprior(p)}
\item \texttt{curve(dposterior(x), from = 0, to = 1, col = "red", add = TRUE, n = 1000)}
\end{itemize}
\item note: this is a toy example, so we really could just use this 
         function to calculate the mean, minimal loss estimator, etc.
\begin{itemize}
\item \texttt{mass <- integrate(function(p) dposterior(p, x, n), lower = 0.25, upper = 0.75)}
\item expected value: \texttt{integrate(function(p) p * dposterior(p, x, n) / mass\$value, lower = 0.25, upper = 0.75)}
\item minimum loss: \texttt{optimize(f = function(phat) integrate(function(p) L2(p - phat) * dposterior(p, x, n) / mass\$value, lower = 0.25, upper = 0.75)\$value, lower = 0.25, upper = 0.75)}
\item \textbf{but} in bigger problems (more parameters, more complicated distributions), these
           direct solutions won't be feasible\ldots{} they'll take too long.
\item metropolis hastings can still work in those bigger problems
\end{itemize}
\end{itemize}
\item to implement, initialize vector of draws (``burn'' first 1000, use second 1000)
\begin{itemize}
\item \texttt{nsim <- 2000}
\item \texttt{Z <- c(rprior(), rep(NaN, nsim))}
\end{itemize}
\item make function that generates $Z_i$ given $Z_{i-1}$
\begin{itemize}
\item \texttt{rZ <- function(Zprev, rv, fv, fy) \{}
\begin{itemize}
\item \texttt{V <- rv()}
\item \texttt{U <- runif(1)}
\item \texttt{probV <- fy(V) * fv(Zprev) / (fv(V) * fy(Zprev))}
\item \texttt{if (U < probV) return(V)}
\item \texttt{else return(Zprev)\}}
\end{itemize}
\end{itemize}
\item run mcmc
\begin{itemize}
\item \texttt{for (i in seq\_len(nsim) + 1)}
\begin{itemize}
\item \texttt{Z[i] <- rZ(Z[i-1], rv = rprior, fv = dprior, fy = function(p) dposterior(p, x, n))}
\end{itemize}
\end{itemize}
\item get posterior:
\begin{itemize}
\item \texttt{hist(Z[-(1:1000)], 40)}
\item \texttt{curve(dposterior(x)/mass\$value, from = 0, to = 1, col = "red", add = TRUE, n = 1000)}
\end{itemize}
\item get estimates:
\begin{itemize}
\item expected value: \texttt{mean(Z[-(1:1000)])}
\item minimum loss: \texttt{optimize(function(phat) mean(L2(Z[-(1:1000)] - phat)), lower = 0.25, upper=0.75)}
\end{itemize}
\item look at the actual draws of $Z$
\begin{itemize}
\item \texttt{plot(Z[1500:1700], type = "o")}
\item really dependent (which makes sense), ie not iid draws
\item doesn't matter, because there is weak enough dependence that we
         can still get averages, etc.
\end{itemize}
\item what happens if our prior doesn't actually contain the ``true'' $p$?
\begin{itemize}
\item draw $x$
\begin{itemize}
\item \texttt{x <- rbinom(1, n, .9)}
\end{itemize}
\item reset $Z$ and rerun mcmc (same as before)
\begin{itemize}
\item \texttt{Z <- c(rprior(), rep(NaN, nsim))}
\item \texttt{for (i in seq\_len(nsim) + 1)}
\begin{itemize}
\item \texttt{Z[i] <- rZ(Z[i-1], rv = rprior, fv = dprior, fy = function(p) dposterior(p, x, n))}
\end{itemize}
\end{itemize}
\item \texttt{hist(Z[-(1:1000)], 40)}
\item bunches up at edge of support
\end{itemize}
\end{itemize}
\section{Bayesian modeling for OLS}
\label{sec-4}

\begin{itemize}
\item Covered on \textit{2010-11-16 Tue}
\item For OLS, we have to specify a joint prior for
\begin{itemize}
\item $\sigma$ (ie $E(\varepsilon_{i}^{2}\mid X)$)
\item $\beta$
\item For convenience, we'll specify a prior for $\beta$
        conditional on $\sigma^2$.
\end{itemize}
\item We'll discuss conjugate and noninformative priors
\item Need to specify distribution of the random variables (Y): $N(X\beta,
      \sigma^{2} I)$ given $X$
\begin{itemize}
\item We'll continue to condition on $X$ (assume X and $\beta$ and
        $\gamma$ are independent).
\end{itemize}
\end{itemize}
\subsection{densities for beta}
\label{sec-4-1}

\begin{itemize}
\item Finite sample theory and asymptotics indicate that
       $\hat{\beta}$ is going to behave as though it is approximately normal.
\item Suggests that getting a normal posterior might be a good idea
\item suggests that starting with a normal prior (for congugate) is too.
\item Prior :: $p(\beta \mid 1/\sigma^{2}) = N(\beta_{0}, \sigma^2 \Sigma_{0})$
\begin{itemize}
\item ie prior is normal with known mean and variance
\end{itemize}
\end{itemize}
\paragraph{posterior distribution}
\label{sec-4-1-1}

\begin{itemize}
\item Use Bayes' rule to get posterior:
        \[ p(\beta \mid X, Y, \sigma^{2}) = \frac{p(Y \mid
        X, \beta, \sigma^{2}) p(\beta \mid X,
        \sigma^{2})}{p(Y \mid X, \sigma^{2})}\]
\item $Y$ given $X$, $\beta$, and $\sigma^{2}$ is
        (obviously) normal
\begin{itemize}
\item mean $X\beta$
\item variance $\sigma^{2} I$
\end{itemize}
\item posterior is
        \[p(\beta \mid X, Y, \sigma^2) \propto (\frac{1}{\sqrt{2 \pi
        \sigma^2}})^n \exp(-\frac{1}{2\sigma^2}(Y - X\beta)'(Y -
        X\beta)) \times (\frac{1}{\sqrt{2^k \pi^k \det(\sigma^2 \Sigma_0)}})
        \exp(- \frac{1}{2} (\beta - \beta_0)'(\sigma^2
        \Sigma_0)^{-1} (\beta - \beta_0))\]
\begin{itemize}
\item terms inside the exponentials can be written as ($\hat\sigma^2$ and
          $\hat \beta$ are the MLE estimators)
          \[\exp(- \frac{1}{2 \sigma^2} \{(\hat \varepsilon - X(\beta -
          \hat\beta))' (\hat \varepsilon - X(\beta - \hat\beta))\]
\begin{itemize}
\item \[(\beta - \beta_0)'\Sigma_0^{-1}(\beta - \beta_0)\})\]
\end{itemize}
which can be rewritten as 
          \[\exp(- \frac{n \hat\sigma^2}{2 \sigma^2}) \exp(-
          \frac{1}{2\sigma^2} \{(\beta - \hat\beta)'X'X(\beta -
          \hat\beta) + (\beta - \beta_0)\Sigma_0^{-1}(\beta - \beta_0)\})\]
\item To get this into something more useful, we want to rewrite
          \[(\beta - \hat\beta)'X'X(\beta -
          \hat\beta) + (\beta - \beta_0)\Sigma_0^{-1}(\beta - \beta_0)
          = (\beta - a \beta_0 - b \hat\beta)'V(\beta - a \beta_0 - b \hat\beta)\]
\begin{itemize}
\item expand both sides and match to find $a$, $b$, and $V$, giving
\begin{itemize}
\item $\beta'V\beta = \beta'X'X\beta + \beta \Sigma_0^{-1} \beta$
\begin{description}
\item[→] $V = X'X + \Sigma_0^{-1}$
\end{description}
\item $\beta'V a \beta_0 = \beta'\Sigma_0^{-1}\beta_0$
\begin{description}
\item[→] $V a = \Sigma_0^{-1}$
\item[→] $a = V^{-1} \Sigma_0^{-1} = (X'X + \Sigma_0^{-1})^{-1} \Sigma_0^{-1}$
\end{description}
\item $\beta'V b \hat\beta = \beta'X'X \hat\beta$
\begin{description}
\item[→] $V b = X'X$
\item[→] $b = V^{-1} X'X = (X'X + \Sigma_0^{-1})^{-1} X'X = I -
                a$
\end{description}
\end{itemize}
\item so we have 
            \[\beta'X'X\beta + \beta \Sigma_0^{-1} \beta = (\beta - a
            \beta_0 - (I - a) \hat\beta)'(X'X + \Sigma_0^{-1}) (\beta - a
            \beta_0 - (I - a) \hat\beta)\]
            with $a = (X'X + \Sigma_0^{-1})^{-1}\Sigma_0^{-1}$
\item the posterior of $\beta$ is $N(a \beta_0 + (I - a)\hat\beta,
            \sigma^2 (X'X + \Sigma_0^{-1})^{-1})$
\end{itemize}
\end{itemize}
\end{itemize}
\paragraph{interpretation from a classical perspective}
\label{sec-4-1-2}

\begin{itemize}
\item We can get a classical estimator by using the posterior mean:
        \[ \hat{\beta}_{bayes} = a \beta_{0} + (I - a)\hat{\beta} \]
\item shrinkage estimator
\begin{itemize}
\item ridge is a special case with $\beta_0 = 0$ and $\Sigma_0 \propto I$
\end{itemize}
\item Can think of the posterior as an average of the prior and MLE
\begin{itemize}
\item Will be biased (in general)
\item will have smaller variance than the OLS estimator.
\end{itemize}
\item In general, as sample size increases and as certainty in
        prior decreases, the ``Bayesian'' estimator behaves more like the
        OLS estimator
\end{itemize}
\paragraph{uninformative prior}
\label{sec-4-1-2-1}

\begin{itemize}
\item $\Sigma_{0}$ denotes how strong our beliefs are
\begin{itemize}
\item small value indicates that we are very confident in our
           prior mean
\item large value indicates that we don't have a lot of
           confidence in our prior mean
\end{itemize}
\item As $\Sigma_{0} \to \infty$, $\Sigma_{0}^{-1}$ converges to zero
\begin{itemize}
\item mean of posterior converges to $\hat{\beta}$
\item variance of posterior converges to $\sigma^{2}(X'X)^{-1}$
\end{itemize}
\item ie, the Bayesian estimator converges to OLS as the prior
         becomes less informative
\item interpretation makes sense: when you don't have strong
         beliefs, you should weight the data heavily
\end{itemize}
\paragraph{asymptotics}
\label{sec-4-1-2-2}

\begin{itemize}
\item As $n \to \infty$, $X'X \to \infty$ and $a \to 0$
\item Same as with uninformative prior; bayesian estimator
         converges to OLS estimator as $n$ gets large
\item makes sense: when you have a lot of data, you shouldn't rely
         too heavily on your prior beliefs.
\end{itemize}
\subsection{densities for sigma-squared}
\label{sec-4-2}

\begin{itemize}
\item Conjugate prior for $1/\sigma^{2}$ is the gamma distribution,
       so $\sigma^2$ has an inverse-gamma distribution:
       \[p_{\sigma^2}(\sigma^2 \mid \alpha, \delta) =
       \frac{\delta^\alpha}{\Gamma(\alpha)}
       (\frac1{\sigma^2})^{\alpha+1} e^{-\frac{\beta}{\sigma^2}},\quad
       \alpha,\delta > 0\]
\begin{itemize}
\item mean is $\delta / (\alpha - 1)$ for $\alpha > 1$
\item variance is $\beta^2 / (\alpha-1)^2(\alpha-2)$ for $\alpha > 2$
\end{itemize}
\item Joint prior for $\beta$ and $1/\sigma^{2}$ is called the
       Normal-Gamma prior:
       \[p_{\beta,\sigma^2}(\beta, \sigma^2 \mid \Sigma_0, \alpha,
       \delta) = (\frac{1}{\sqrt{2^k \pi^k \det(\sigma^2 \Sigma_0)}})
       \exp(- \frac{1}{2} (\beta - \beta_0)'(\sigma^2
       \Sigma_0)^{-1} (\beta - \beta_0))
       \times        \frac{\delta^\alpha}{\Gamma(\alpha)}
       (\frac1{\sigma^2})^{\alpha+1} e^{-\frac{\delta}{\sigma^2}}\]
\item We can work out the posterior as before:
       \[p(\beta, \sigma^2 \mid X, Y) \propto (\frac{1}{\sqrt{2 \pi
        \sigma^2}})^n \exp(-\frac{1}{2\sigma^2}(Y - X\beta)'(Y -
        X\beta)) \quad\times (\frac{1}{\sqrt{2^k \pi^k \det(\sigma^2 \Sigma_0)}})
        \exp(- \frac{1}{2} (\beta - \beta_0)'(\sigma^2
        \Sigma_0)^{-1} (\beta - \beta_0)) \quad\times        \frac{\delta^\alpha}{\Gamma(\alpha)}
       (\frac1{\sigma^2})^{\alpha+1} \exp(-\frac{\delta}{\sigma^2})
       =  (\frac{1}{\sqrt{2 \pi \sigma^2}})^n  (\frac{1}{\sqrt{2^k \pi^k
       \det(\sigma^2 \Sigma_0)}}) \exp(- \frac{n \hat\sigma^2}{2
       \sigma^2}) \quad \times \exp(-
          \frac{1}{2\sigma^2} \{(\beta - \hat\beta)'X'X(\beta -
          \hat\beta) + (\beta - \beta_0)\Sigma_0^{-1}(\beta -
       \beta_0)\})\quad\times        \frac{\delta^\alpha}{\Gamma(\alpha)}
       (\frac1{\sigma^2})^{\alpha+1} \exp(-\frac{\delta}{\sigma^2})  \]
       which is proportional to 
       \[ (\frac{1}{2 \pi \sigma^2})^{k/2} \det(X'X + \Sigma_0^{-1})
       \exp( -\frac{1}{2\sigma^2} (\beta - a \beta_0 - (I -
       a)\hat\beta)' \times (X'X + \Sigma_0^{-1})(\beta - a \beta_0 - (I - a)\hat\beta))
        \times (\frac{1}{2 \pi \sigma^2})^{\frac{n}{2}}
       \exp(- \frac{n\hat\sigma^2}{2\sigma^2})       \frac{\delta^\alpha}{\Gamma(\alpha)}
       (\frac1{\sigma^2})^{\alpha+1} \exp(-\frac{\delta}{\sigma^2})\]
\begin{itemize}
\item The first part is the posterior of $\beta$ conditional on $\sigma^2$.
\item We can rewrite the second part to be more interpretable:
\item combining terms and dropping scale constants gives:
         \[(\frac{1}{\sigma^2})^{n/2 + \alpha + 1} \exp(- \frac{\hat\sigma^2 n/2 +
         \delta}{\sigma^2})\]
\item after thinking for a bit, we can see that this is the kernel of an inverse
         gamma with parameters
\begin{itemize}
\item $n/2+\alpha$ and
\item $\hat\sigma^2 n/2 + \delta$
\end{itemize}
\item mean is \[\frac{\hat\sigma^2 n/2 + \delta}{n/2 + \alpha - 1}
         = \hat\sigma^2 \frac{n}{n + 2\alpha - 2} +
         \delta(\frac{2}{n + 2\alpha - 2})\]
\item to interpret this,
\begin{itemize}
\item define prior mean: $\sigma_0^2 = \frac{\delta}{\alpha - 1}$
\item $\delta = \sigma^2_0 (\alpha - 1)$
\item posterior mean becomes 
           \[\hat\sigma^2 \frac{n}{n + 2\alpha - 2} +
           \sigma_0^2(\frac{2 \alpha - 2}{n + 2\alpha - 2}) =
           w \hat\sigma^2 + (1-w) \sigma_0^2\]
\item Bayes estimate is a weighted average of the MLE and the
           prior mean
\item as $n \to \infty$, $w \to 1$ and MLE dominates
\item we can write prior variance as $\sigma_0^4 / (\alpha - 2)$
\begin{itemize}
\item for $\sigma_0$ fixed, as $\alpha \to \infty$ prior
             variance converges to zero and $w \to 0$ as well, so
             posterior mean dominates.
\item for $\sigma_0$ fixed, as $\alpha \to 1$ from above, $w$
             converges to 1 again and MLE dominates
\end{itemize}
\end{itemize}
\end{itemize}
\end{itemize}
\subsection{putting it all together}
\label{sec-4-3}

     After the analysis above, we can write the posterior density of
     $\beta$ and $\sigma^2$ explicitly as
     \[ (\frac{1}{2 \pi \sigma^2})^{k/2} \det(X'X + \Sigma_0^{-1})
     \exp( -\frac{1}{2\sigma^2} (\beta - a \beta_0 - (I -
     a)\hat\beta)' \times (X'X + \Sigma_0^{-1})(\beta - a \beta_0 - (I - a)\hat\beta))
      \times \frac{(\hat\sigma^2 n/2 + \delta)^{\alpha +
     n/2}}{\Gamma(n/2 + \alpha)} (\frac{1}{\sigma^2})^{n/2 + \alpha +
     1} \exp(-\frac{\hat\sigma^2 n/2 + \delta}{\sigma^2}) \]

%%% Local Variables:
%%% mode: latex
%%% TeX-master: "../../additionaltopics"
%%% End:

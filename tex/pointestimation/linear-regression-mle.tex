% Copyright © 2013, Gray Calhoun.  Permission is granted to copy,
% distribute and/or modify this document under the terms of the GNU
% Free Documentation License, Version 1.3 or any later version
% published by the Free Software Foundation; with no Invariant
% Sections, no Front-Cover Texts, and no Back-Cover Texts.  A copy of
% the license is included in the section entitled "GNU Free
% Documentation License."

\part*{Linear regression and MLE}%
\addcontentsline{toc}{part}{Linear regression and MLE}

\begin{itemize}
\item cover on
\begin{itemize}
\item \textit{2009-09-30 Wed}, \textit{2009-10-14 Wed}, \textit{2009-10-19 Mon}
\item \textit{2010-09-28 Tue}, \textit{2010-10-21 Thu}
\item \textit{2011-09-13 Tue}, \textit{2011-10-27 Thu}
\item \textit{2012-09-11 Tue}, \textit{2012-09-13 Thu}
\end{itemize}
\end{itemize}
\section{Introduction}
\label{sec-1}

\begin{itemize}
\item look at the uniform$(0,b)$ example suppose $n = 1$
\begin{itemize}
\item draw these densities for b = 1, 2, 4
\begin{itemize}
\item density is $1/b$ in $[0,b]$.
\item pick a point $x$ on the density
\end{itemize}
\item have (as a function of $b$) $1/b$ as long as $b \geq x$
\item if we observe $X = 0.5$ (for example), we know that $b \geq 0.5$
\item In addition, values like $b = 500$ are implausible
\begin{itemize}
\item as $b$ gets larger, there are more possible values that $X$ is
         likely to take on.
\end{itemize}
\item in a sense, $b = 0.5$ is the most plausible
\begin{itemize}
\item smaller values are impossible
\item as $b$ increases from $0.5$, the plausiblity decreases.
\end{itemize}
\item we can get $b = 0.5$ directly by maximizing $f(0.5; b)$ as a
       function of $b$
\begin{itemize}
\item called the likelihood and written $L(b; x)$
\end{itemize}
\end{itemize}
\end{itemize}
\section{Definition}
\label{sec-2}

\begin{itemize}
\item for the MLE estimator, we start with the density, but view it as a
     function of $\theta$
\begin{itemize}
\item the value of $\theta$ that maximizes the likelihood is (in a sense) the most plausible/defensible value.
\end{itemize}
\item the MLE of $\theta$ is $argmax_\theta L(\theta; x_1,\dots,x_n)$
\end{itemize}
\section{Examples}
\label{sec-3}
\subsection{iid draws from uniform$(a,b)$}
\label{sec-3-1}

\begin{itemize}
\item $L(a,b; x_1,\dots,x_n) = \prod_i 1\{x_i \in [a,b]\} (b-a)^{-1}$
\item $= ({1 \over b-a})^{n}$ if all $x_i \in [a,b]$ and zero otherwise.
\item we can find the maximum easily
\begin{itemize}
\item likelihood decreases as $b$ increases or $a$ decreases
\item likelihood becomes zero if $b < \max x_i$ or if $a > \min x_i$
\item so $\hat b = \max x_i$ and $\hat a = \min x_i$
\end{itemize}
\end{itemize}
\subsection{linear regression.}
\label{sec-3-2}

\begin{itemize}
\item $(Y_i,X_i) \sim iid$
\begin{itemize}
\item $X_i \sim f$ which is unspecified, $X_i$ is a $k\times1$ vector.
\item $Y_i \mid X_i \sim N(X_i'\beta, \sigma^2)$
\end{itemize}
\item want to estimate $\beta$ and $\sigma^2$
\item Draw fitted values, densities
\item $L(\mu,\sigma^2; y, x) = \prod_i {1 \over \sqrt{2 \pi \sigma^2}} e^{- {(y_i - x_i'\beta)^2 \over 2 \sigma^2}} f(x_i)$
\item step 1: take logs:
\begin{itemize}
\item $\log L(\mu,\sigma^2; x_1,\dots,x_n) = const - n\log
         (\sqrt{\sigma^2}) - \sum_i {(x_i - \mu)^2 \over 2 \sigma^2} + \sum_i f(x_i)$
\end{itemize}
\item First order conditions:
\begin{itemize}
\item for mean:
\begin{itemize}
\item $\frac{\partial}{\partial \beta} \log L(\mu, \sigma^2; x, y) = \sum_{i=1}^n x_i (y_i - x_i'\beta) = 0$
\item so $\hat\beta=(\sum_i x_i x_i')^{-1} \sum_i x_i y_i$
\end{itemize}
\item for variance:
\begin{itemize}
\item $\frac{\partial}{\partial \sigma^2} \log L(\mu, \sigma^2; x, y) = -\frac{n}{2\sigma^2} + \frac{1}{2 \sigma^4}\sum_i (y_i - x_i'\beta)^2 = 0$
\item so $\hat\sigma^2 = \frac{1}{n} \sum_{i=1}^n (y_i - x_i'\beta)^2$
\end{itemize}
\end{itemize}
\item students should verify that this is a maximum on their own.
\end{itemize}
\section{More remarks on mle}
\label{sec-4}

\begin{itemize}
\item unlike method of moments, where we connect our parameters to
      the mean, variance, etc. regardless of the distribution; here
      we look at the features of the data that the distribution tells
      us are the most relevant.
\begin{itemize}
\item for normal, this \emph{is} the mean and variance, so MLE
        and MoM give us the same statistics
\item for uniform, and others, this is \emph{not} the mean and
        variance.     - the derivative of the log likelihood is called the \emph{score}.
        $S(\theta; x) = {\partial \over \partial \theta} \log L(\theta; x)$
\end{itemize}
\item has a nice invariance property: say you're not interested in the
      parameters per se, but care about a transformation of the
      parameters $T(\theta)$.  If $\hat \theta$ is the maximum
      likelihood estimator of $\theta$, then $T(\hat\theta)$ is the MLE
      of $T(\theta)$.
\item we'll see later that you can use MLE to get an estimator, even if
      you don't believe that the distribution is true
\begin{itemize}
\item called quasi-maximum likelihood
\item obviously, you then need to check that the estimator works well.
\end{itemize}
\end{itemize}

%%% Local Variables: 
%%% mode: latex
%%% TeX-master: "../../pointestimation"
%%% End: 

% Copyright © 2013, Gray Calhoun.  Permission is granted to copy,
% distribute and/or modify this document under the terms of the GNU
% Free Documentation License, Version 1.3 or any later version
% published by the Free Software Foundation; with no Invariant
% Sections, no Front-Cover Texts, and no Back-Cover Texts.  A copy of
% the license is included in the section entitled "GNU Free
% Documentation License."

\part*{Linear regression and Method of Moments}%
\addcontentsline{toc}{part}{Linear regression and the Method of Moments}

\begin{itemize}
\item Cover on
\begin{itemize}
\item \textit{2009-09-28 Mon}
\item \textit{2010-09-28 Tue}
\item \textit{2011-09-08 Thu}, \textit{2011-09-13 Tue}
\item \textit{2012-09-11 Tue}
\end{itemize}
\item Reading
\end{itemize}
\section{Description}
\label{sec-1}

\begin{itemize}
\item Suppose we have $X_1,\dots,X_n \sim f(x; \theta)$
\begin{itemize}
\item $f$ is known
\item $\theta$ is an unknown $p$-vector that we want to estimate
\end{itemize}
\item estimator is based on a simple idea:
\begin{itemize}
\item if we want to estimate an expectation, use a sample average
\item $P[X_i \leq c]$ for known $c$
\begin{itemize}
\item $=E 1\{X_i \leq x\}$
\item estimate with $n^{-1} \sum_{i=1}^n 1\{X_i \leq c\}$
\end{itemize}
\item Estimate $\E X_i$ with $n^{-1} \sum_{i=1}^n X_i$
\item etc
\end{itemize}
\item in general, relate $\theta$ to the population moments:
\begin{itemize}
\item if $\theta$ has $p$ elements, we calculate the first $p$ moments
        of $X_i$
\begin{itemize}
\item $\mu_1 = g_1(\theta)$
\item $\mu_2 = g_2(\theta)$
\item dot dot dot
\item $\mu_p = g_p(\theta)$
\end{itemize}
\end{itemize}
\item calculate the sample moments:
\begin{itemize}
\item $\hat \mu_1 = n^{-1}\sum_{i=1}^n X_i$
\item $\hat \mu_2 = n^{-1}\sum_{i=1}^n X_i^2$
\item dot dot dot
\item $\hat \mu_p = n^{-1}\sum_{i=1}^n X_i^p$
\end{itemize}
\item estimate $\theta$ by setting the equations equal:
\begin{itemize}
\item $g_1(\hat\theta) = \hat\mu_1 = n^{-1}\sum_{i=1} X_i$
\item dots
\item $g_p(\hat\theta) = \hat\mu_p = n^{-1}\sum_{i=1} X_i^p$
\item then $\hat\theta = g^{-1}(n^{-1}\sum_{i=1}^n
        X_i,\dots,n^{-1}\sum X_i^p)$
\begin{itemize}
\item obviously, $g$ needs to be invertible for this to work.
\end{itemize}
\end{itemize}
\item Sometimes $\hat\theta$ is easy to write out analytically as a
      function of the sample averages
\begin{itemize}
\item when it's not, you can find $\hat \theta$ numerically.
\end{itemize}
\item if each sample moment is close to the population moment and
      $g^{-1}$ is continuous, then $\hat\theta$ should be close to $\theta$.
\end{itemize}
\section{Examples}
\label{sec-2}

\begin{itemize}
\item normal$(\mu,\sigma^2)$
\begin{itemize}
\item first moment of a normal random variable is $E X_i = \mu$
\item second moment is $E X_i^2 = \sigma^2 + \mu^2$
\item method of moments estimator is
\begin{itemize}
\item $\hat \mu = n^{-1} \sum_i X_i$
\item $\hat \sigma^2 = n^{-1} \sum_i X_i^2 - (n^{-1}\sum_i X_i)^2
           = n^{-1} \sum_i(X_i - \bar X)^2$
\item which is not the usual estimator, but is close.
\end{itemize}
\end{itemize}
\item uniform(a,b)
\begin{itemize}
\item Let's say that $X_1,\dots,X_n$ are iid uniform$(a,b)$, and we want
         to calculate the method of moments estimator for $a$ and $b$.
\begin{itemize}
\item density of $X_i$ is $1/(b-a)$ in $[a,b]$, zero otherwise.
\end{itemize}
\item Calculate the first two moments:
\begin{itemize}
\item $E X_i = \int_a^b x /(b-a) dx = \frac{b+a}{2}$
\item $E X_i^2 = \int_a^b x^2 /(b-a) dx = \frac{b^3 - a^3}{3(b - a)} = \frac{b + ab + a^2}{3}$
\end{itemize}
\item Gives the estimators:
\begin{itemize}
\item $\hat b+\hat a = 2 n^{-1} \sum_i X_i$
\item ${\hat b^3-\hat a^3 \over (\hat b- \hat a)} = 3 n^{-1} \sum_i X_i^2$
\item solve for $\hat b$ and $\hat a$, gives
\begin{itemize}
\item $\hat b = \bar X + s \sqrt{3}$
\item $\hat a = \bar X - s \sqrt{3}$
\item $s = \sqrt{n^{-1} \sum_i (X_i - \bar X)^2}$
\end{itemize}
\end{itemize}
\end{itemize}
\item linear regression
\begin{itemize}
\item setup
\begin{itemize}
\item \$(Y$_i$, X$_i$) $\sim$ i.i.d.\$
\item $X_i \sim f$ (unspecified)
\item $Y_i \mid X_i \sim N(\beta_0 + \beta_1 X_i, \sigma^2)$
\item want to estimate $\beta_0$ and $\beta_1$
\item draw scatterplot (this is like estimating the slope and intercept)
\item We'll often see this as $Y_i = \beta_0 + \beta_1 X_i + \varepsilon_i$
\begin{itemize}
\item $\varepsilon_i \mid X_i \sim N(0, \sigma^2)$
\item i.e. define $\varepsilon_i = Y_i - \beta_0 - \beta_1 X_i$
\end{itemize}
\end{itemize}
\item MoM estimator
\begin{itemize}
\item we know $E \binom{Y_i}{X_i Y_i} = \begin{pmatrix} \beta_0 + \beta_1 E X_i \\ \beta_0 E X_i + \beta_1 E X_i^2 \end{pmatrix}$
\begin{itemize}
\item $= \begin{pmatrix} 1 & E X_i \\ E X_i & E X_i^2 \end{pmatrix} \begin{pmatrix}\beta_0 \\ \beta_1\end{pmatrix}$
\end{itemize}
\item Assuming invertibility, $\begin{pmatrix}\beta_0\\\beta_1\end{pmatrix} = \begin{pmatrix}1&EX_i\\EX_i&EX_i^2\end{pmatrix}^{-1} \begin{pmatrix}EY_i\\EX_iY_i\end{pmatrix}$
\item This makes our estimator, $\begin{pmatrix}\hat\beta_0\\\hat\beta_1\end{pmatrix}=\begin{pmatrix}1&\sum_i x_i/n\\\sum_i x_i/n&\sum_ix_i^2/n\end{pmatrix}^{-1}\begin{pmatrix}\sum_iy_i/n\\\sum_ix_iy_i/n\end{pmatrix}$
\begin{itemize}
\item $=\begin{pmatrix}1&\sum_i x_i\\\sum_i x_i&\sum_ix_i^2\end{pmatrix}^{-1}\begin{pmatrix}\sum_iy_i\\\sum_ix_iy_i\end{pmatrix}$
\item (write in matrix notation and explain)
\end{itemize}
\end{itemize}
\end{itemize}
\end{itemize}
\section{Discussion}
\label{sec-3}

\begin{itemize}
\item advantages
\begin{itemize}
\item it gets you an estimator
\item easy to derive asymptotic properties of the estimator (it's a
         function of averages, which usually obey CLTs and LLNs).
\end{itemize}
\item disadvantages
\begin{itemize}
\item can be very inefficient
\item for it to be useful, requires that the moments tell you a lot
         about the distribution (for the uniform, they don't
         necessarily -- we're trying to estimate the endpoints, but our
         estimator is to look for the center and double it.)
\begin{itemize}
\item a big problem with our estimates here, is we might have \$X\$s
           in our data that are outside the interval $[a,b]$
\end{itemize}
\item kind of ad hoc.
\item for the uniform distribution, we have \emph{all} of the moments
         defined\ldots{} should we calculate the first $p$ moments and
         average all of them?
\begin{itemize}
\item how do we pick $p$?
\item are some moments better than others?
\end{itemize}
\end{itemize}
\end{itemize}
\section{GMM}
\label{sec-4}

\begin{itemize}
\item Hansen (1982) shows that a modification of the method of
         moments can be very useful in macro:
\begin{itemize}
\item have periods $t=1,\dots,T$
\item macro models tell you that $E_t g(X_t, \theta) = 0$ where
           $g(x_t, \theta)$ is coming from Euler equations (ie, an
           agent is optimizing in period $t$, and $g$ captures the
           difference between what they expect to happen in period
           $t+1$ and what actually happens in period $t+1$ -- under
           rationality, they choose an action that makes that
           difference unpredictable.
\item LIE tells you that $E g(X_t, \theta) = E E_t g(X_t,\theta)$
           which equals zero, so you have the condition
           \[E g(X_t, \theta) = 0\]
\item Hansen shows that you can often estimate $\theta$ by solving
           \[ T^{-1} \sum_{t=1}^T g(X_t,\theta) = 0 \]
           for $\theta$. (and says how)
\item when you have more moments than parameters, gives a
           weighting scheme.
\end{itemize}
\end{itemize}

%%% Local Variables: 
%%% mode: latex
%%% TeX-master: "../../pointestimation"
%%% End: 

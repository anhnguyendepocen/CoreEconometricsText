% Copyright © 2013, authors of the "Econometrics Core" textbook; a
% complete list of authors is available in the file AUTHORS.tex.

% Permission is granted to copy, distribute and/or modify this
% document under the terms of the GNU Free Documentation License,
% Version 1.3 or any later version published by the Free Software
% Foundation; with no Invariant Sections, no Front-Cover Texts, and no
% Back-Cover Texts.  A copy of the license is included in the file
% LICENSE.tex and is also available online at
% <http://www.gnu.org/copyleft/fdl.html>.

\part*{Asymptotic evaluation of point estimators}%
\addcontentsline{toc}{part}{Asymptotic evaluation of point estimators}

\section{Asymptotic Efficiency}

\begin{itemize}
\item To talk about efficiency, we need to restrict our attention to
      asymptotically normal estimators
\begin{itemize}
\item rules out pathalogical counterexamples
\item I should add one next year (homework?)
\item kind of makes sense; for normal r.v. only the mean and
        variance matter; for other asymp. distributions other
        parameters are more important.
\end{itemize}
\item If $\hat\theta_n$ satisfies $m_n (\hat\theta_n - \theta) \to^d
      N(0, \sigma^2)$ for some (increasing) sequence $m_n$ then
      $\sigma^2$ is the asymptotic variance of $\hat\theta_n$.
\item Suppose that $X_1,\dots,X_n$ have the likelihood fn $L(\theta;
      X)$ and that $\hat\theta_n$ is a sequence of estimators of
      $\theta$.  $\hat\theta_n$ is asymptotically efficient for
      $\theta$ if $m_n (\hat\theta_n - \theta) \to^d N(0, \sigma^2)$
      and 
      \[\sigma^2 = \Big[E_\theta ((\tfrac{\partial}{\partial \theta} \log L(\theta; X))^2)\big]^-1\].
\begin{itemize}
\item i.e. $\hat\theta_n$ achieves the CR lower bound in the limit.
\end{itemize}
\item Nice result: under regularity conditions (like we saw before),
      MLEs are consistent and asymptotically normal as well as
      efficient.
\end{itemize}

%%% Local Variables: 
%%% mode: latex
%%% TeX-master: "../../pointestimation"
%%% End: 

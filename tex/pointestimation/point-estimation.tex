% Copyright © 2013, Gray Calhoun.  Permission is granted to copy,
% distribute and/or modify this document under the terms of the GNU
% Free Documentation License, Version 1.3 or any later version
% published by the Free Software Foundation; with no Invariant
% Sections, no Front-Cover Texts, and no Back-Cover Texts.  A copy of
% the license is included in the section entitled "GNU Free
% Documentation License."

\part*{Overview of point estimation}%
\addcontentsline{toc}{part}{Overview of point estimation}

\begin{itemize}
\item Covered on
\begin{itemize}
\item \textit{2009-09-23 Wed}, \textit{2009-09-28 Mon}
\item \textit{2010-09-23 Thu}
\item \textit{2011-09-08 Thu}
\end{itemize}
\item we're not learning probability for its own sake
\begin{itemize}
\item we actually want to talk about statistical estimators
\item the probability tools that we've learned are going to help.
\end{itemize}
\end{itemize}
\section{datasets}
\label{sec-1}

\begin{itemize}
\item we want to think about datasets as a particular instance of an
      unobserved random process.
\item Most datasets have the form (and I'll do a macro dataset)
\begin{description}
\item[time period] \ldots{},2009:01, 2009:02, 2009:3,\ldots{}
\item[unemployment] \ldots{},7.6, 8.1, 8.5,\ldots{}
\item[CPI] 212.174, 213.007, 212.714,\ldots{}
\end{description}
\item we can think of this as a sequence of 2-vectors
\begin{itemize}
\item t will be the index (if we have data starting in 1970, t would
        be 1 in Jan of 1970, 2 in Feb of 1970),\ldots{},468 in 2009:01, 469
        in 2009:02, etc.
\item $x_t$ will be the \emph{observed values} of unemployment and cpi in
        month t
\begin{itemize}
\item $x_{468} = (7.6, 212.174)$
\item $x_{469} = (8.1, 213.007)$
\item etc.
\end{itemize}
\end{itemize}
\item So, this is the data -- $x_1,\dots,x_T$
\item Working assumption
\begin{itemize}
\item data were generated by some sort of hypothetical experiment (DGP).
\item we want to use the data to estimate features of the DGP
\end{itemize}
\item DGP might be $X_t \sim N(\mu, \sigma^2)$; the $x_t$ are
      realizations of $X_t$ and we want to estimate $\mu$ and
      $\sigma^2$.
\item or $\Delta p_t = g(\Delta p_{t-1}, \Delta p_{t-2}, \dots; u_{t-1}, u_{t-2}, \dots) + \varepsilon_t$;
\begin{itemize}
\item $\varepsilon_t$ is independent of $\Delta p_{t-1}, \Delta p_{t-2}, \dots$ and $u_{t-1}, u_{t-2}, dots$
\item $g$ is smooth
\item want to estimate $\frac{\partial}{\partial u_{t-1}} g$
\end{itemize}
\item The DGP we write down doesn't need to be true
\begin{itemize}
\item some statistics will be more sensitive than others to how ``true'' the model is.
\end{itemize}
\end{itemize}
\section{definition of estimators}
\label{sec-2}

    Let $x_1,\dots,x_n$ be a dataset.  A function of that dataset,
    $T(x_1,\dots,x_n)$ is called a \emph{statistic} or \emph{estimator}.
\begin{itemize}
\item examples:
\begin{itemize}
\item $T(x_1,\dots,x_n) = 1/2$
\item $T(x_1,\dots,x_n) = n^{-1} \sum_{i=1}^n x_i$
\item histogram
\end{itemize}
\item Can't depend on the true DGP
\begin{itemize}
\item $T(x_1,\dots,x_n) = E X_1$ is not an estimator
\end{itemize}
\item so a statistic just takes the numbers you see, and summarizes
      them.  Anything that does that is called a statistic.
\begin{itemize}
\item even if the ``summary'' is stupid.  for example, if you
        ``summarize'' the numbers with 0, it is a statistic (as far as the
        theory is concerned).
\item The ``summary'' can also be random.  We won't get into that too
        much this class.
\end{itemize}
\item So, there are lots of things that are bad or uninformative that
      we're going to treat as statistics/estimators.
\begin{itemize}
\item But that's okay, because we haven't developed any way to say
        that an estimator is ``bad'' or ``good'' yet.
\item until we do, we can't say that summarizing a dataset with the
        mean of each of its variables is ``good'' and summarizing the
        dataset with the number 0 is ``bad''.
\item moreover, letting ``bad'' things count as estimators helps us
        define what ``good'' means.
\end{itemize}
\item we are talking about the \textbf{function}, not the value it takes on.
\begin{itemize}
\item estimators are random variables
\item the realization is an \emph{estimate}.
\end{itemize}
\end{itemize}

%%% Local Variables: 
%%% mode: latex
%%% TeX-master: "../../pointestimation"
%%% End: 

% Copyright © 2013, authors of the "Econometrics Core" textbook; a
% complete list of authors is available in the file AUTHORS.tex.

% Permission is granted to copy, distribute and/or modify this
% document under the terms of the GNU Free Documentation License,
% Version 1.3 or any later version published by the Free Software
% Foundation; with no Invariant Sections, no Front-Cover Texts, and no
% Back-Cover Texts.  A copy of the license is included in the file
% LICENSE.tex and is also available online at
% <http://www.gnu.org/copyleft/fdl.html>.

\part*{Multivariate random vectors}%
\addcontentsline{toc}{part}{Multivariate random vectors}

 The borel sigma field on $\Re^k$, denoted $\mathbb{B}^k$ is the
     smallest sigma-field containing the clos.ed
     rectangles: \[\{[a_1,b_1] \times \cdots \times [a_k,b_k] : -
     \infty \leq a_i \leq b_i \leq \infty\}\]

\section{multivariate distributions}
\subsection{marginal distribution}

\begin{itemize}
\item since a random vector is just a vector of random variables, each
       element has its own distribution
\item these are called ``marginal distributions''
\end{itemize}

\subsection{joint distribution}

\begin{itemize}
\item different elements can depend on each other
\item this dependence isn't reflected in the marginal distributions
\item is reflecteded in the ``joint distribution'' of the entire vector
\end{itemize}

\paragraph{definition}
      The joint distribution, $F_X$, of the $k$-dimensional random
      vector $X$ is define as the function 
      \[ F_X(x_1,\dots,x_k) = P(X_1 \leq x_1,\dots,X_k \leq x_k) \]

\paragraph{relationship between joint and marginal}
\begin{itemize}
\item marginal distributions tell you less about the random vector
\item can get marginals from joint, but not vice-versa.
\begin{itemize}
\item different joint distributions can have the same marginal
          distributions.
\end{itemize}
\item formula (for 1st marginal, similar for others)
        \[ F_{X_1}(x) = F_X(x, \infty, \dots,\infty) \]
\begin{itemize}
\item follows immediately from the definition of both sides
          \[ P(X_1 \leq x) = P(X_1 \leq x, X_2 \leq \infty, \dots, X_k
          \leq \infty) \]
\end{itemize}
\item to summarize:
\begin{itemize}
\item the joint distribution determines the marginals
\item the marginals do not determine the joint (maybe do for hw)
\end{itemize}
\end{itemize}

\section{multivariate densities}
\subsection{joint and marginal distributions and densities}

      Same as before: we want to define weights on points, not just on
      regions.

\paragraph{definitions}
\begin{itemize}
\item marginal densities are the densities of the individual elements
\item joint densities describe the simultaneous behavior of all of
        the elements
\begin{itemize}
\item pmf: The probability mass function of a discrete random vector,
          denoted $f_X(x_1,\dots, x_k)$, is given by 
          \[f_X(x_1,\dots,x_k) = P[X_1 = x_1,\dots,X_k = x_k]\]
\item pdf: the probability density function of a continuous random
          variable is the function $f_X(\cdot)$ that satisfies
          \[ F_X(c_1,\dotsc,c_k) = \int_{-\infty}^{c_1} \dots
          \int_{-\infty}^{c_k} f(y_1,\dots,y_k) dy_1 \dots dy_k \]
\end{itemize}
\end{itemize}

\paragraph{properties}
\begin{itemize}
\item usual properties hold
\begin{itemize}
\item densities are positive
\item densities integrate up to 1.
\end{itemize}
\item can get marginal density from joint
        \[ f_{X_1}(x) = \int f_X(x, y_2,\dots,y_k) dy_2 \dots dy_k \]
\begin{itemize}
\item follows from writing distribution in terms of integral of
          the density
\end{itemize}
\end{itemize}

\subsection{transformations of random vectors}

\begin{itemize}
\item a general theory is very painful to work with (particularly
        because of dimension changes)
\item The textbook presents the result for bivariate transformations
\item The extension to higher-dimension random vectors is
        straightforward
\end{itemize}

\paragraph{Jacobian}
      Let $g: \Re^k \to \Re^k$ be a differentiable function.  The
      Jacobian of this function is the matrix of partial derivatives
      \[ J(x) = \begin{pmatrix} \partial g_1(x) / \partial x_1 & \partial
      g_1(x) / \partial x_2 & \cdots & \partial g_1 / \partial x_k \\
      \vdots \\ \partial g_k / \partial x_1 & \cdots & \cdots & \partial g_k /
      \partial x_k \end{pmatrix} \]

\paragraph{result \citep[B.7.7]{Gre_2011}}
      Suppose that $X$ is a random vector in $\Re^k$ with joint
      density $f_X$ and that $g: \Re^k \to \Re^k$ is one-to-one from
      the support of $X$ to its image and let $Y = g(X)$.  Then the
      joint density of $Y$ is \[ f_Y(y) = f_X(g^{-1}(y)) abs(det( J(y)
      ))\] (where $J(y)$ is the Jacobian of $g^{-1}$) in the image of
      the support of $X$ and is zero elsewhere.
\begin{itemize}
\item the restriction that $g$ be one-to-one can be relaxed.
\end{itemize}

\paragraph{transformations to lower dimension}
      the requirement that $g$ be from $\Re^k$ to $\Re^k$ is restrictive.
      For transformations from a higher dimension to a lower dimension,
      it's best to create an artificial transformation that's easy to
      work out, then integrate out the other dimension.

\subsection{independence}

\begin{itemize}
\item remember how we defined independence for events
\begin{itemize}
\item Two events $A_1$ and $A_2$ are independent if $P(A_1 \cap
         A_2) = P(A_1) \times P(A_2)$.
\end{itemize}
\item same idea for random variables, but now events are intervals in $\Re$
\begin{itemize}
\item Two random variables $X_1$ and $X_2$ are \emph{independent} if \[
         P(X_1 \in I_1 \cap X_2 \in I_2) = P(X_1 \in I_1) P(X_2 \in
         I_2)\] for any intervals $I_1$ and $I_2$.
\end{itemize}
\item implications:
\begin{itemize}
\item Two random variables with marginal distributions $F_1$ and
         $F_2$ and joint distribution function $F_X$ are independent iff
         $F_X(x_1, x_2) = F_1(x_1) F_2(x_2)$
\item Two random variables with marginal densities $f_1$ and $f_2$
         and joint density $f_X$ are independent iff 
         $f_X(x_1,x_2) = f_1(x_1) f_2(x_2)$
\item if $X_1$ and $X_2$ are independent, so are $g(X_1)$ and
         $h(X_2)$ for any (measurable) functions $g$ and $h$
\begin{itemize}
\item proof:
\begin{itemize}
\item $P[g(X_1) \in I_1 \cap h(X_2) \in I_2]$
\begin{description}
\item[.] $= P[X_1 \in g^{-1}(X_1) \cap X_2 \in h^{-1}(X_2)]$
\item[.] $= P[X_1 \in g^{-1}(X_1)] P[X_2 \in h^{-1}(X_2)]$
\item[.] $= P[g(X_1) \in I_1] P[h(X_2) \in I_2]$
\end{description}
\item here \[g^{-1}(I_1) = \{x : g(x) \in I_1\},\] we don't
             need $g$ to be invertible.
\end{itemize}
\end{itemize}
\end{itemize}
\end{itemize}

\section{Moments}
\subsection{Expectation for a random vector}

\begin{itemize}
\item $\E X = (\E X_1,\dots,\E X_k)$
\item $\E g(X_1,\dots,X_k) = \int g(x_1,\dots,x_k) f_X(x_1,\dots,x_k)
       dx_1 \cdots dx_k$
\item If $X$ and $Y$ are independent with finite expectations, $\E XY
       = \E X \E Y$ (proof automatic).
\end{itemize}

\subsection{second moments}

\paragraph{Variance}
\begin{itemize}
\item for vectors, $var(X) = E(X - EX) (X - EX)'$
\begin{itemize}
\item this is positive semi definite
\begin{itemize}
\item for any vector $\alpha$, $\alpha'var(X)\alpha \geq 0$
\item proof: $\alpha' var(X) \alpha = alpha' E((X - EX)(X - EX)') \alpha$
\begin{itemize}
\item $= E( alpha'(X - EX)(X - EX)' \alpha)$
\item $= E( (alpha'X - E(alpha'X))^2)$
\item $= var(alpha'X) \geq 0$
\end{itemize}
\end{itemize}
\item Let $X$ be a random $k$-vector with vcv $\Sigma$ and let
         $A$ be a deterministic $j \times k$ matrix.  Then $A X$
         has vcv $A \Sigma A'$.
\end{itemize}
\end{itemize}

\paragraph{Covariance}
\begin{itemize}
\item $cov(X,Y) = E(X - E X) (Y - EY) = E XY - EX EY$
\begin{itemize}
\item obviously, $cov(X,X) = var(X)$
\end{itemize}
\item diagonal elements of vcv matrix of $X$ are the variances of individual elements of $X$
\item we can see that the off-diagonal elements are the covariances
\item for vectors $X$ and $Y$, $cov(X,Y) = E(X - EX)(Y - EY)'$
\item $var(X)$ is positive semi definite
\begin{itemize}
\item i.e. for any nonzero vector $\alpha$, $\alpha' var(X)
            alpha \geq 0$
\item proof: $\alpha'var(X) \alpha = \alpha' E((X - EX)(X -
            EX)') \alpha$
\begin{itemize}
\item $= E(\alpha'(X - EX)(X - EX)' \alpha)$
\item $= E((\alpha'(X-EX))^2) \geq 0$
\end{itemize}
\end{itemize}
\end{itemize}

\paragraph{Correlation}
\begin{itemize}
\item Definition: $corr(X, Y) = cov(X,Y) / \sqrt{var(X)} \sqrt{var(Y)}$
\item this is between -1 and 1, which will be an implication of the Cauchy-Schwarz inequality
\end{itemize}

\paragraph{Cauchy-schwarz inequality}
\begin{itemize}
\item Let $X$ and $Y$ be random variables.  Then $E (|X Y|) \leq
          \sqrt{E X^2} \sqrt{E Y^2}$
\item we're going to show that this is equivalent to something
          like $0 \leq something^2$
\item proof:
\begin{itemize}
\item we want to show $(E |XY|)^2 \leq E X^2 E Y^2$
\item equivalent to $0 \leq E X^2 - (E |X Y|)^2 / E Y^2$ (as long
            as $Y$ is not identically zero)
\item $= E X^2 - 2 (E |XY|)^2/EY^2 + (E |XY|)^2/EY^2$
\item $= E X^2 - 2(E |XY| / E Y^2) E |XY| + (E |XY|/EY^2)^2 E Y^2$
\item $= E(X^2 - 2(E |XY| / E Y^2) |X| |Y| + (E |XY|/EY^2)^2 Y^2)$
\item $= E(|X| - (E |XY|/E Y^2) |Y|)^2$
\end{itemize}
\end{itemize}

%%% Local Variables: 
%%% mode: latex
%%% TeX-master: "../../probability"
%%% End: 
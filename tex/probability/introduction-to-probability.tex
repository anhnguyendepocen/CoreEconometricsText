% Copyright © 2013, Gray Calhoun.  Permission is granted to copy,
% distribute and/or modify this document under the terms of the GNU
% Free Documentation License, Version 1.3 or any later version
% published by the Free Software Foundation; with no Invariant
% Sections, no Front-Cover Texts, and no Back-Cover Texts.  A copy of
% the license is included in the section entitled "GNU Free
% Documentation License."

\part*{Introduction to the probability lectures}%
\addcontentsline{toc}{part}{Quick overview of the probability material}

\begin{itemize}
\item This motivation is largely taken from Resnick's \emph{A Probability      Path}.  There are a few key theoretical concepts that lead to
     many econometric tools.
\item covered on \textit{2009-08-24 Mon}, \textit{2010-08-24 Tue}, \textit{2011-08-23 Tue},
     \textit{2012-08-23 Thu}
\end{itemize}
\section{Law of Large Numbers}
\label{sec-1}

     Suppose that $X_1,\cdots,X_n$ is a sequence of independent,
     identically distributed (iid from here on) random variables with
     mean $\mu$.  Then $\bar X \to \mu$ as $n \to \infty$
\begin{itemize}
\item average payoff at a 25-cent slot machine gets close to the
       expected payoff as I play for longer
\item haven't specified what the arrow denotes formally
\item justifies using the average payoff as an estimator of expected
       payoff
\end{itemize}
\section{Central Limit Theorem}
\label{sec-2}

     If $X_1,\cdots,X_n$ are iid with mean $\mu$ and variance $\sigma$$^2$
     then
     \[ P[n^{1/2}(\bar X - \mu)/\sigma \leq c] \to \Phi(c)
      \equiv \int_{-\infty}^c (2\pi)^{-1/2} e^{-u^2/2} du
     \]
\begin{itemize}
\item this is a stronger result than the law of large numbers
\item I can use this result to find out the probability of winning more
       than (say) \$15 at a slot machine after playing for a long time.
\end{itemize}

%%% Local Variables: 
%%% mode: latex
%%% TeX-master: "../../probability"
%%% End: 
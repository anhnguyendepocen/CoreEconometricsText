% Copyright © 2013, Gray Calhoun.  Permission is granted to copy,
% distribute and/or modify this document under the terms of the GNU
% Free Documentation License, Version 1.3 or any later version
% published by the Free Software Foundation; with no Invariant
% Sections, no Front-Cover Texts, and no Back-Cover Texts.  A copy of
% the license is included in the section entitled "GNU Free
% Documentation License."

\part*{Univariate Moments}%
\addcontentsline{toc}{part}{Univariate moments}

\section{basic expectation}

\subsection{introduction and motivation}

\begin{itemize}
\item one view of the steps we've taken
\begin{itemize}
\item We've proposed random variables as a way to simplify experiments
         in a way that doesn't leave out any ``essential'' details.
\item density functions let us simplify further
\item continuing this goal, we're going to see how we can simplify even further.
\end{itemize}
\item ways to simplify
\begin{itemize}
\item what single number is a random draw from the given distribution
         going to be ``closest'' to ``on average''
\begin{itemize}
\item obviously, this depends on what we mean by ``closest''
\item different meanings will give us different summary values
\end{itemize}
\item what is the most likely value that an r.v. will take on
\end{itemize}
\end{itemize}

\subsection{definition}

\begin{itemize}
\item The mean (or expected value) of a random variable, X, denoted
       E X, is defined for continuous r.v. by \[ E X =
       \int_{-\infty}^{\infty} x f_X(x) dx \]
\item for discrete rv, \[ E X = \sum_x x f_X(x) dx \]
\item for random vectors, $E Y = (E Y_1, \dots, E Y_k)$
\end{itemize}

\subsection{interpretation}

\begin{itemize}
\item population analogue to sample average (and limit of sample
       average usually)
\item point where the densities would balance
\item ``closest'' to a random draw on average, in the sense that
       $E(X - E X)^2 \leq E(X - c)^2$ for any point c.
\end{itemize}

\subsection{other points}

\begin{itemize}
\item if $E|X| = \infty$, we say that the mean does not exist.
\begin{itemize}
\item happens if the density has ``fat tails''
\end{itemize}
\item the mean is sometimes written as $EX = \int_{-\infty}^{\infty} x
       dF(x)$ since $dF/dx = f$
\item basic linearity: $E(a X + b) = a E X + b$ for constants $a$
       and $b$
\item The expectation of $Y = g(X)$ is given by \[ E g(X) =
       \int_{-\infty}^\infty g(x) f_X(x) dx \] which is a much
       easier way to go than to calculate the density
       of the transformed rv.
\begin{itemize}
\item holds for random vectors too
\end{itemize}
\item as a consequence, we have the result: If $X$ and $Y$ are
       independent then $E XY = EX E Y$
\begin{itemize}
\item $EXY = \int\int x y f_{X,Y}(x,y) dx dy$
\begin{itemize}
\item $= \int\int xy f_X(x) f_Y(y) dx dy$
\item $= \int x f_X(x) dx \int y f_Y(y) dy$
\item $= EX EY$
\end{itemize}
\end{itemize}
\end{itemize}

\subsection{Jensen's inequality}

\begin{itemize}
\item a function $g(x)$ is convex if $g(\lambda x + (1-\lambda)y)
       \leq \lambda g(x) + (1-\lambda) g(y)$ for $\lambda \in [0,1]$
\item For any random variable $X$, $\E g(X) \geq g(\E X)$ for any
       convex $g$.
\end{itemize}

\section{other measures of location}

\begin{itemize}
\item the mean is (by far) the most common way to summarize a random variable.
\item there are others
\end{itemize}

\subsection{median}

\begin{itemize}
\item Def: the median of a distribution F is the value c such that $P(X
       \leq c) \geq 1/2$ and $P(X \geq c) \geq 1/2$.
\item the median is not necessarily unique
\item if X is continuous, the median satisfies 
       \[ E | X - c | \geq E | X - median(X) | \]
       for any other constant $c$; so this is another way of chosing
       ``the closest value to the average draw'' for a different
       interpretation of close.
\begin{itemize}
\item the mean weights values further from the center more than
         the median
\end{itemize}
\item you can define the conditional median same as conditional expectation
\end{itemize}

\subsection{mode}

\begin{itemize}
\item Def: the mode of a distribution is the value $c$ that maximizes
       the density function $f_X$
\item the mode is not necessarily unique
\item conditional mode exisits too
\end{itemize}
     measures of location are unhelpful without measures of uncertainty

\section{second moments (variance)}

\begin{itemize}
\item measures ``how far'' an random draw is from its expected value
\item definition: $var(X) = E(X - E X)^2$
\begin{itemize}
\item equivalently, $var(X) = E(X^2) - (EX)^2$
\item obviously this is nonnegative
\end{itemize}
\item standard deviation is the square root of the variance
\begin{itemize}
\item this is in the same units as $X$
\end{itemize}
\end{itemize}

\section{location scale transformations}

\begin{itemize}
\item Useful special case of the transformation formula:
\item Let $X$ be a random variable with density $f_X$ and let $f$ be
      the density of $\sigma X + \mu$ for some constant $\mu$ and
      positive constant $\sigma$.  Then \[f(x) = \frac{1}{\sigma}
      f_X\left(\frac{x - \mu}{\sigma}\right)\] for all $x$.
\begin{itemize}
\item This family of pdfs is called the ``location-scale family
        with standard pdf $f(x)$".
\item $\mu$ is the \emph{location parameter}
\item $\sigma$ is the \emph{scale parameter}
\end{itemize}
\item The family $\{f_X(x - \mu) : \mu \in \mathbb{R}\}$ is a
      ``location family'' (i.e. only $\mu$ varies)
\item The family $\{(1/\sigma) f_X(x/\sigma) : \sigma > 0\}$ is a
      ``scale family'' (i.e. only \$$\sigma$ varies)
\item $1/\sigma f_X(\frac{x - \mu}{\sigma})$ is a pdf for any density
      $f_X$ and constants $\mu$ and $\sigma > 0$.
\end{itemize}

\section{other moments}

\begin{itemize}
\item If $X$ is a random variable, we say that the kth moment of $X$
      exists if $E|X|^k$ is finite
\item skewness: (draw picture)
\begin{description}
\item[Greene's definition] $E(X - E X)^3$
\item[more common] $E(X - E X)^3 / \sigma^3$
\end{description}
\item kurtosis: (draw picture; move mass from center to tails)
\begin{description}
\item[Greene's definition] $E(X - E X)^4$
\item[more common] $E(X - E X)^4 / \sigma^4$
\item[excess] $E(X - E X)^4 / \sigma^4 - 3$
\end{description}
\item in general, we might look at powers like these
\begin{itemize}
\item kth moment: $E X^k$
\item kth central moment: $E(X - EX)^k$
\end{itemize}
\item Consequence of Jensen's inequality: if the \$p\$th moment is
      finite, so are all moments between 0 and $p$.
\begin{itemize}
\item Proof: let $q \in (0, p)$.  \[\E |X|^p = \E[( |X|^q)^{p/q}]
        \leq [\E(|X|^q)]^{p/q}\] since $p/q > 1$.  Since the last
        quantity is finite, so is the first.
\end{itemize}
\end{itemize}

%%% Local Variables: 
%%% mode: latex
%%% TeX-master: "../../probability"
%%% End: 
% Copyright © 2013, Gray Calhoun.  Permission is granted to copy,
% distribute and/or modify this document under the terms of the GNU
% Free Documentation License, Version 1.3 or any later version
% published by the Free Software Foundation; with no Invariant
% Sections, no Front-Cover Texts, and no Back-Cover Texts.  A copy of
% the license is included in the section entitled "GNU Free
% Documentation License."

\chapter{Hypothesis testing}

\begin{itemize}
\item Covered on
\begin{itemize}
\item \textit{2009-09-30 Wed}, \textit{2009-10-05 Mon}
\item \textit{2010-09-30 Thu}
\item \textit{2011-09-20 Tue}, \textit{2011-09-22 Thu}, \textit{2011-09-27 Tue},
       \textit{2011-09-29 Thu}
\end{itemize}
\item Reading
\begin{itemize}
\item CB 8.1, 8.2.3, 8.3.1, 8.3.3-8.3.5
\item Greene C.7
\end{itemize}
\end{itemize}
\section{Introduction}
\label{sec-1}

\begin{itemize}
\item Motivation
\begin{itemize}
\item there are many settings where we care what the unknown
       coefficients are.
\begin{itemize}
\item forecasting
\item setting policy
\end{itemize}
\item there are plenty of other settings where we don't really care what
       the coefficients are, but just want to know whether the
       coefficient is above some threshold
\end{itemize}
\item Example: have a basic relationship: $wage_t = \beta_0 + \beta_1
     \times education_t + \varepsilon_t$
\begin{itemize}
\item may want to know whether $\beta_1$ is positive
\item may want to know whether $\beta_1$ is above some threshold that
       makes funding a program worthwhile
\item probably don't care about whether or not $\beta_1$ is exactly
       2.95 vs 2.97 -- they probably mean exactly the same thing as far
       as your decision is concerned.  But $-0.01$ vs 0.01 probably don't
       mean exactly the same thing as far as decisions are concerned
\item For (most) estimation, though, we would treat mismeasuring the
       true parameter by 0.02 the same, whether the true value was 2.95
       or $-0.01$
\end{itemize}
\end{itemize}
\section{Basic setup for testing}
\label{sec-2}
\subsection{Definition of test statistic}
\label{sec-2-1}

\begin{itemize}
\item Suppose that you have two different hypotheses about the
       unknown parameters of the DGP
\begin{description}
\item[null] $\theta \in \Theta_0$
\begin{itemize}
\item the null hypothesis is what we would believe about $\theta$ if we
           had no data about the problem
\end{itemize}
\item[alternative] $\theta \in \Theta_a$
\begin{itemize}
\item the alternaive hypothesis is a candidate belief that might be true.
\item but we need \textbf{strong} evidence before we believe it
\end{itemize}
\end{description}
\item A \emph{test statistc} is a statistic that specifies the following:
\begin{enumerate}
\item For which samples do we make the decision to accept $\theta_a$
          (our alternative) as true
\item For which samples do we \textbf{not} make that decision, and instead
          act as though $\theta_0$ were true.
\end{enumerate}
\end{itemize}
\subsection{Example of a test statistic}
\label{sec-2-2}

\begin{itemize}
\item Suppose $X_1,\dots,X_n \sim iid N(\mu,1)$ and we have the hypotheses:
\begin{description}
\item[null] $\mu = 0$
\item[alt] $\mu = 1$
\end{description}
\item we can use the standard one-sided test statistic:
\begin{description}
\item[reject 0] if $\sqrt{n} (\bar X - \mu_0) / \sigma = \sqrt{n}
                     \bar X \geq 1.64$
\item[accept 0] if $\sqrt{n} \bar X < 1.64$
\end{description}
\item tells us that we reject null for any sample that gives us a mean
       greater than $1.64 / \sqrt{n}$ and do not reject for other
       samples
\end{itemize}
\subsection{Error types and probabilities}
\label{sec-2-3}

\begin{itemize}
\item Test statistics can't be foolproof.
\begin{itemize}
\item Even if $\mu = 0$, we'll see some samples that have a sample
         mean greater than $1.64 / \sqrt{n}$
\item If $\mu = 1$, we'll also see samples with a small mean
\item The key aspect of statistical testing is that we can calculate
         the probabilities of making different sorts of errors, and
         design procedures that make errors that we're comfortable with.
\end{itemize}
\item Possible outcomes:
\begin{itemize}
\item null is true, test statistic does not reject
\begin{itemize}
\item great!
\end{itemize}
\item null is true, test statistic does reject
\begin{itemize}
\item problem
\end{itemize}
\item null is false, test statistic does not reject
\begin{itemize}
\item problem
\end{itemize}
\item null is false, test statistic rejects
\begin{itemize}
\item great!
\end{itemize}
\end{itemize}
\item key philosophy behind testing:
\begin{itemize}
\item we set up the problems so that the null is what we'd do
         without data.
\item consequently, if the null is false, but the test statistic
         does not reject, it's not so bad
\item \textbf{but} if the null is true and we reject it, that is bad.
\begin{itemize}
\item we should need more evidence to change our behaviour than we
           would need to keep doing the same thing we were.
\end{itemize}
\item advantages of this approach:
\begin{itemize}
\item simplifies the mathematics considerably
\item intuitively appealing.
\item matches up with actual practice in ``academic'' scientific experiments
\end{itemize}
\end{itemize}
\item terminology
\begin{itemize}
\item type one error: reject when you shouldn't
\item type two error: don't reject when you should
\item power: probability that the test rejects (usually will depend
         on the value of the true parameter).
\begin{itemize}
\item formally, let $\beta(\theta) = P_\theta[reject]$.  $\beta(\theta)$ is the power function.
\end{itemize}
\item size: probability that the test rejects if the null hypothesis is true;
\begin{itemize}
\item i.e. probability of type I error
\item a test is \emph{size} $\alpha$ if $\sup_{\theta \in \Theta_0} \beta(\theta) = \alpha$
\item a test is \emph{level} $\alpha$ if $\sup_{\theta \in \Theta_0} \beta(\theta) \leq \alpha$
\item \textbf{size} is what we care most about.  We want to set the size
\end{itemize}
\item valid: a test that has correct size -- ie rejects if the null
         is true at the percentages we want
\item unbiased: a test is unbiased if $\inf_{\theta \in \Theta_A} \beta(\theta) \geq \alpha$
\end{itemize}
\end{itemize}
\subsection{Motivating example}
\label{sec-2-4}

\begin{itemize}
\item sometimes we can monkey with a statistic to get a test
\item think back to normal distribution with $H_0: \mu = 0$ vs
       $H_a: \mu = 1$ and we want to test at 5\%
\begin{itemize}
\item statistic was $\sqrt{n} \bar X$.
\item if $\mu = 0$, then we know that $\sqrt{n} \bar X$ is standard normal.
\item $\bar X > 0$ is evidence against the null, so let's try to
         find a threshold that preserves size
\begin{itemize}
\item find $c$ such that $\Pr[\sqrt{n} \bar X \geq c] = 0.05$
\item $\Pr[\sqrt{n} \bar X \geq c] = 1 - \Pr[\sqrt{n} \bar X < c]$
\begin{itemize}
\item $= 1 - \Pr[\sqrt{n} \bar X \leq c]$ (continuity)
\item $= 1 - \Phi(c)$
\end{itemize}
\item so, $\Phi(c) = 0.95$ and $c = \Phi^{-1}(0.95) \approx 1.64$
\end{itemize}
\item So the probability that $\sqrt{n} \bar X \geq 1.64 = 0.05$
\item so, the probability of rejecting if the null hypothesis is
         true is 0.05.
\begin{itemize}
\item the size of the test is 0.05
\item if we want to change the size of the test, all we have to
           do is change 1.64 to a different number.
\end{itemize}
\end{itemize}
\item power
\begin{itemize}
\item knowing the distribution under the null gives us size
\item to find out the power, we want to know the distribution under
         the alternative
\begin{itemize}
\item if $\mu = 1$ then $\sqrt{n}(\bar X - 1)$ is standard normal.
\item $\beta(1) = P[\sqrt{n} \bar X \geq 1.64] = P[\sqrt{n}(\bar
           X - 1) \geq 1.64 - \sqrt{n}] = 1 - \Phi(1.64 - \sqrt{n}) \to 1$
\end{itemize}
\end{itemize}
\end{itemize}
\subsection{p-values}
\label{sec-2-5}

\begin{itemize}
\item Say you're working, do a test at 5\% and want to show your advisor
\item maybe your advisor is more or less tolerant of type 1 error
\item so you test at 10\%, 9.9\%, 9.8\%, etc and record rejections and acceptances
\item but, you could summarize that by looking at the smallest critical value for which you would reject
\item this critical value is the p-value
\end{itemize}
\section{Multiple comparisons}
\label{sec-3}

    Some special issues come up when we have many hypotheses.  I need
    to figure out some simulations to motivate this stuff.
\subsection{Intersection-Union Test}
\label{sec-3-1}

\begin{itemize}
\item Suppose the null can be written as an intersection of simpler nulls:
\begin{itemize}
\item $H_0: \theta \in \bigcup_{\gamma \in \Gamma} \Theta_\gamma$ vs. 
         $H_A: \theta \notin \bigcap_{\gamma \in \Gamma} \Theta_\gamma$
\item It is easy to find tests for each individual set:
\begin{itemize}
\item $T_\gamma$ tests the null $\theta \in \Theta_\gamma$
           vs. $\theta \notin \Theta_\gamma$ for each $\gamma$.
\end{itemize}
\item If $T_\gamma$ is a level-$\alpha$ test with rejection region
         $R_\gamma$ (ie it rejects if $(x_1,\dots,x_n) \in R$), then
         the test with rejection region $\bigcap_{\gamma \in \Gamma}
         R_\gamma$ is the IU Test and has level $\alpha$
\end{itemize}
\item Note that we don't need to correct the critical values here
\end{itemize}
\paragraph{Proof of validity}
\label{sec-3-1-1}

\begin{itemize}
\item We know that, under the null, $\theta \in \Theta_\gamma'$ for at least one $\gamma'$
\item $\Pr_\theta[(X_1,\dots,X_n) \in \bigcap_\gamma R_\gamma] \leq \Pr_\theta[(X_1,\dots, X_n) \in R_{\gamma'}] \leq \alpha$
\end{itemize}
\subsection{Union-Intersection Test}
\label{sec-3-2}
\paragraph{Setup of UIT}
\label{sec-3-2-1}

\begin{itemize}
\item Suppose the null can be written as an intersection of simpler nulls:
\begin{itemize}
\item $H_0: \theta \in \bigcap_{\gamma \in \Gamma} \Theta_\gamma$ vs. 
         $H_A: \theta \notin \bigcap_{\gamma \in \Gamma} \Theta_\gamma$
\item It is easy to find tests for each individual set:
\begin{itemize}
\item $T_\gamma$ tests the null $\theta \in \Theta_\gamma$
           vs. $\theta \notin \Theta_\gamma$ for each $\gamma$.
\end{itemize}
\item If $T_\gamma$ is a level-$\frac{\alpha}{\# \Gamma}$ test with
         rejection region $R_\gamma$ (ie it rejects if
         $(x_1,\dots,x_n) \in R$), then the test with rejection region
         $\bigcup_{\gamma \in \Gamma} R_\gamma$ is the UI Test and has
         level $\alpha$
\end{itemize}
\item Note that we don't need to use a stepdown procedure here
\item Could still do better if we used the joint distribution of $T_\gamma$
\end{itemize}
\paragraph{Show that testing at $\alpha$ doesn't work}
\label{sec-3-2-2}
\paragraph{Proof of validity}
\label{sec-3-2-3}

\begin{itemize}
\item Follows from the same argument as in multiple comparisons
\item $\Pr_\theta[(X_1,\dots,X_n) \in \bigcup_{\gamma} R_\gamma] \leq \sum_{\gamma \in \Gamma} \Pr_\theta[(X_1,\dots,X_n) \in R_\gamma]$
\begin{itemize}
\item $\leq \sum_\gamma \frac{\alpha}{\#\Gamma}$
\item $= \alpha$
\end{itemize}
\end{itemize}
\subsection{Multiple hypothesis testing}
\label{sec-3-3}

\begin{itemize}
\item Suppose we have a bunch of hypotheses $\theta \in
       \Theta_\gamma$, but instead of being interested in knowing
       about compound hypotheses, we're interested in all of them
\begin{itemize}
\item have many different parameters
\item Put up empirical example where we're looking at regressors
\end{itemize}
\item Now we want to control the rate of ``familywise error''
\begin{itemize}
\item Let $I$ index the true nulls: $I = \{\gamma \in \Gamma : \theta_\gamma \in \Theta_\gamma\}$
\item FWE is \[\sup_{\theta: \theta_\gamma \in \Theta_\gamma, \gamma \in I} P_\theta[T_\gamma \text{ rejects for at least one } \gamma \in I]\]
\item want this error probability to be less than $\alpha$
\end{itemize}
\item As with UIT, testing each one at $\alpha$ doesn't work
\begin{itemize}
\item show this
\end{itemize}
\end{itemize}
\paragraph{Bonferroni bound}
\label{sec-3-3-1}
\paragraph{Bonferroni-Holm stepdown}
\label{sec-3-3-2}
\paragraph{Critical values from the joint distribution}
\label{sec-3-3-3}
% Copyright © 2013, Gray Calhoun.  Permission is granted to copy,
% distribute and/or modify this document under the terms of the GNU
% Free Documentation License, Version 1.3 or any later version
% published by the Free Software Foundation; with no Invariant
% Sections, no Front-Cover Texts, and no Back-Cover Texts.  A copy of
% the license is included in the section entitled "GNU Free
% Documentation License."

\part{Asymptotic inference}

\begin{itemize}
\item Covered on
\begin{itemize}
\item \textit{2010-10-05 Tue}
\item \textit{2011-10-18 Tue}
\end{itemize}
\end{itemize}
\section{Introduction and definitions}
\label{sec-1}

\begin{itemize}
\item One problem with the testing results we've done so far
\begin{itemize}
\item need to know the DGP (up to unknown constants) to get the
        distribution of the statistic.
\begin{itemize}
\item An example would be nice (maybe X1+X2 from normal vs. uniform)
\end{itemize}
\item this is unrealistic in practice
\end{itemize}
\item CLT suggests taht, for a lot of DGPs and statistics, the exact
      distribution isn't going to matter much, as long as we have a
      lot of observations
\item So we can use the CLT to derive an approximate distribution for
      our statistics, and use approximate critical values.
\item Definition: let $\{T_n\}$ be a sequence of test statistics for
      $\theta \in \Theta_0$ against $\theta \in \Theta_A$.  The
      sequence of tests has asymptotic size $\alpha$ if 
      \[\lim_{n \to \infty} \sup_{\theta \in \Theta_0} \Pr_\theta[T_n \text{ rejects}] = \alpha\]
\begin{itemize}
\item asymptotic level if the equality is an inequality.
\end{itemize}
\item The same idea holds for confidence intervals
\begin{itemize}
\item Let $\{[L_n, U_n\}$ be a sequence of interval estimators for a
        parameter $\theta$.  The asymptotic confidence level for this
        sequence of statistics is
        \[\lim_{n\to\infty} \inf_\theta \Pr_{\theta}[\theta \in [L_n, U_n]]\]
\item important that the limit is outside the inf or sup
\item you should think of a good question or example to address that
\begin{itemize}
\item maybe from delta method?
\end{itemize}
\end{itemize}
\end{itemize}
\section{Wald Test}
\label{sec-2}

\begin{itemize}
\item The Wald tests is built on the CLT explicitly
\end{itemize}
\subsection{Main idea}
\label{sec-2-1}

\begin{itemize}
\item Suppose we want to test the null $\theta = \theta_0$ and we have
      an asymptotically normal estimator $\hat\theta_n$: $\sqrt{n}
      (\hat\theta_n - \theta) \to^d N(0,\eta^2)$ for all $\theta$.
\item Then we can base a test on $\frac{\sqrt{n}}{\hat\eta}
      (\hat\theta - \theta_0)$ as long as $\hat\eta$ is a consistent
      estimator of $\eta$. (i.e. a $t$-test or something derived from it.
\begin{itemize}
\item often get a chi-squared distribution
\end{itemize}
\end{itemize}
\subsection{Example}
\label{sec-2-2}

\begin{itemize}
\item Suppose $(Y_i, X_i) \sim i.i.d$ with unknown density
\begin{itemize}
\item $E(Y_i \mid X_i) = \beta_0 + \beta_1 X_i$
\item $var(Y_i \mid X_i) = \sigma^2$
\end{itemize}
\item Want to test the null that $\beta_0 = \beta_1 = 0$ against
       two-sided alternatives.
\begin{itemize}
\item Use our MLE $\hat\beta$ = $(X'X)^{-1} X'Y$
\end{itemize}
\item Step 1, prove that $\sqrt{n}(\hat\beta - \beta) \to N(0, \Sigma = \sigma^2 (E x_i x_i')^{-1})$
\item step 2, prove that $\hat\sigma^2$ converges in prob to $\sigma^2$
\item Step 3, argue that $\frac{n}{\hat\sigma^2}
       \hat\beta'\hat\Sigma^{-1} \hat\beta$ is asymptotically
       chi-square with 2 degrees of freedom.
\item Step 4, explain test statistic.
\end{itemize}
\section{LRT result}
\label{sec-3}

\begin{itemize}
\item The chi-square limit distribution comes up a lot
\item A nice result for the LRT gives us a chi-square limit as well
\item Let $X_1,\dots,X_n \sim i.i.d.\ f(x; \theta)$ adn let
      $\lambda(X)$ be the LR statistic for the test $\theta \in
      \Theta_0$ vs. $\theta \notin \Theta_0$.  Under appropriate
      regularity conditions,
      \[-2 \log \lambda(X) \to^d \chi_p^2\]
      under the null, where $p$ is the number of restrictions imposed
      by the null (in the wald test example, there wre two
      restrictions)
\end{itemize}
\section{Other miscellaneous points}
\label{sec-4}

\begin{itemize}
\item You can also get a test from the first order conditions (called
      the ``score'' or ``LM'' test)
\item results for testing immediately translate into results for
      confidence intervals/sets.
\end{itemize}
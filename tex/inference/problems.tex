% Copyright © 2013, Gray Calhoun.  Permission is granted to copy,
% distribute and/or modify this document under the terms of the GNU
% Free Documentation License, Version 1.3 or any later version
% published by the Free Software Foundation; with no Invariant
% Sections, no Front-Cover Texts, and no Back-Cover Texts.  A copy of
% the license is included in the section entitled "GNU Free
% Documentation License."

\part*{Problem Set}%
\addcontentsline{toc}{part}{Problem Set}

This is a collection of problems for self-study.

\begin{enumerate}

\item Suppose that $X₁,…,X_n$ are distributed uniform$(0,b)$.  Derive
  the LRT for the null hypothesis $b ≥ b₀$ against the alternative $b
  < b₀$ and also for the null hypothesis $b ≤ b₀$ against $b > b₀$.
  Please discuss and compare the tests.

\item Suppose that $X₁,…,X_n$ are i.i.d. uniform($a$,$b$).  Derive the
  LRT of the null hypothesis $\E X_i = 0$ against the two-sided
  alternative $\E X_i ≠ 0$.

\item Suppose that $X₁,…,X_n ∼$ i.i.d. uniform($a$,$b$).
  \begin{enumerate}
  \item Show that $(\min_i X_i, \max_i X_i)$ is the maximum likelihood
    estimator of $(a,b)$.
  \item Prove that the MLE is consistent.
  \item Please find the asymptotic distribution of the MLE.  You will
    need to rescale the estimator to find an asymptotic distribution.
    Two hints: it is not asymptotically normal; and you may need to
    use the result $(1 + \tfrac{1}{n})^n → e$ as $n → ∞$.
  \item Use your answer to the previous question to construct a
    two-sided 90\% confidence interval for $b$.
  \item When we derived the asymptotic distribution earlier, we used
    only some aspects of the assumption that $X_i ∼$ uniform($a$,$b$).
    Can you show that $\max_i X_i$ and $\min_i X_i$ have the same
    asymptotic distribution that we derived above under weaker
    assumptions?
  \end{enumerate}

\item Let $X = (X₁,…,X_n)$ be a random sample and let $θ$ be some
  parameter of interest.  For each $θ₀$, let $A(θ₀)$ be the acceptance
  region of a level $α$ test of the null hypothesis that $θ = θ₀$.
  For each sample $x$, define a set $C(x)$ in the parameter space
  by
  \begin{equation}
     C(x) = \{ θ₀ : x ∈ A(θ₀) \}.
  \end{equation}
  Prove that the random set $C(X)$ is a $1-α$ confidence set for the
  parameter $θ$.

\end{enumerate}
%%% Local Variables: 
%%% mode: latex
%%% TeX-master: "../../inference"
%%% End: 

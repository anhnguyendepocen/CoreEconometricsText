% Copyright © 2013, authors of the "Econometrics Core" textbook; a
% complete list of authors is available in the file AUTHORS.tex.

% Permission is granted to copy, distribute and/or modify this
% document under the terms of the GNU Free Documentation License,
% Version 1.3 or any later version published by the Free Software
% Foundation; with no Invariant Sections, no Front-Cover Texts, and no
% Back-Cover Texts.  A copy of the license is included in the file
% LICENSE.tex and is also available online at
% <http://www.gnu.org/copyleft/fdl.html>.

\part*{Problem set, hypothesis testing}%
\addcontentsline{toc}{part}{Problem set, hypothesis testing}

\begin{enumerate}

\item Let $\{y_t\}$ be an iid $N(0,σ²)$ sequence.  Define $S_T =
  (T-1)^{-1} ∑_{t=1}^T (y_t - \bar y)²$.
  \begin{enumerate}
  \item Prove that $\sqrt{n} (S_T - σ²) → N(0,2σ⁴)$ in
    distribution.
  \item Calculate the asymptotic distribution of $\log S_T$.
  \item Do your results depend on the normality of $y_t$?
  \item Calculate the 90th and 95th percentile for $S_T$ with
    arbitrary values of $σ²$ using both of these asymptotic
    distributions.
  \item Use each of these results to derive a test of the null
    hypothesis $σ² = σ₀²$ against the null $σ² > σ₀²$, where $σ₀²$ is
    a known but arbitrary value.  Your answer should give two formulas
    for the test's critical value---each one depends on $σ₀²$ and $α$
    (the nominal size of the test) and you should have a separate
    answer for each asymptotic approximation.
  \item Let $c₁$ and $c₂$ denote the 90th percentiles that you
    calculated in question ?.  Simulate 1000 i.i.d. standard normal
    samples with $n = 50$ and calculate the probability that $S_T$ is
    less than each of these percentiles.
  \item Repeat the previous question for the 95th percentiles.
  \item Plot a histogram of your 1000 simulated $S_T$ along with each
    approximate density for $S_T$.  What do these results tell you
    about the quality of the approximations?
  \item You can also prove that $(T-1) S_T$ has a chi-square
    distribution with $T-1$ degrees of freedom in finite samples.
    Calculate the 90th and 95th percentiles of $S_T$ using this
    chi-square distribution and repeat the previous two simulations.
    How do these simulations compare to the previous simulations?
  \item Repeat the previous three questions using a skewed
    distribution and a heavy-tailed distribution (changing the
    approximation as necessary).  How do the results change?  How do
    they change if you use different values of $n$?
  \item What do these simulations tell you about using these
    approximations for testing.  Focus on the usual confidence levels
    (i.e. 10\%, 5\%, and 1\% tests).
  \end{enumerate}

\item Suppose that $X₁,…,X_n$ are distributed uniform$(0,b)$.  Derive
  the LRT for the null hypothesis $b ≥ b₀$ against the alternative $b
  < b₀$ and also for the null hypothesis $b ≤ b₀$ against $b > b₀$.
  Please discuss and compare the tests.

\item Suppose that $X₁,…,X_n$ are i.i.d. uniform($a$,$b$).  Derive the
  LRT of the null hypothesis $\E X_i = 0$ against the two-sided
  alternative $\E X_i ≠ 0$.

\item Let $X = (X₁,…,X_n)$ be a random sample and let $θ$ be some
  parameter of interest.  For each $θ₀$, let $A(θ₀)$ be the acceptance
  region of a level $α$ test of the null hypothesis that $θ = θ₀$.
  For each sample $x$, define a set $C(x)$ in the parameter space
  by
  \begin{equation}
     C(x) = \{ θ₀ : x ∈ A(θ₀) \}.
  \end{equation}
  Prove that the random set $C(X)$ is a $1-α$ confidence set for the
  parameter $θ$.

\end{enumerate}

%%% Local Variables: 
%%% mode: latex
%%% TeX-master: "../../inference"
%%% End: 

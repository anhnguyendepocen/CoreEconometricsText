% Copyright © 2013, Gray Calhoun.  Permission is granted to copy,
% distribute and/or modify this document under the terms of the GNU
% Free Documentation License, Version 1.3 or any later version
% published by the Free Software Foundation; with no Invariant
% Sections, no Front-Cover Texts, and no Back-Cover Texts.  A copy of
% the license is included in the section entitled "GNU Free
% Documentation License."

\chapter{Linear Regression}

\begin{itemize}
\item Covered on \textit{2010-11-04 Thu}, \textit{2010-11-09 Tue}
\item \textit{2011-11-10 Thu}, \textit{2011-11-15 Tue}, \textit{2011-11-17 Thu}
\end{itemize}
\section{Estimation under restrictions}
\label{sec-1}

\begin{itemize}
\item This material is based on Greene's 5.1, 5.2, and 5.3.2.
\item Covered on \textit{2010-11-04 Thu}
\end{itemize}
\subsection{introduction}
\label{sec-1-1}

    We've been looking at the model $Y = X\beta + \varepsilon$ or
    (equation by equation) \[ y_i = x_{i,1} \beta_1 + \cdots +
    x_{i,2} \beta_2 + \varepsilon_i \]
\begin{itemize}
\item what if you thought that (say) $\beta_1 = \beta_2$
\item an example (Greene p. 81); have a model of investment \[ \ln I_t
      = \beta_1 + \beta_2 i_t + \beta_3 \Delta p_t + \beta_4 \ln Y_t +
      \beta_5 t + \varepsilon_t \]
\begin{itemize}
\item $I_t$ is investment in period $t$
\item $i_t$ is the nominal interest rate
\item $\Delta p_t$ is the rate of inflation
\item $Y_t$ is real output
\item time trend
\end{itemize}
\item so suppose you think that inflation and the nominal interest rate
      don't matter on their own, but combined they do matter as the
      \emph{real} interest rate
\begin{itemize}
\item model becomes
        \[ \ln I_t = \beta_1 + \beta_2 (i_t - \Delta p_t) + \beta_4 \ln
        Y_t + \beta_5 t + \varepsilon_t \]
\item ie you want to estimate the first model under the restriction
        that $\beta_2 = - \beta_3$.
\item we will rewrite the restriction as $\beta_2 + \beta_3 = 0$.
\end{itemize}
\item So, how would you estimate the first equation under this
      restriction?
\end{itemize}
\subsection{LM restrictions for a simple example}
\label{sec-1-2}

\begin{itemize}
\item to derive the OLS estimators, we minimized $\sum_i (y_i -
      x_i'\beta)^2$
\begin{itemize}
\item this is equivalent to MLE under normality
\end{itemize}
\item If we want to estimate the restricted model, we can just minimize
      $\sum_i (y_i - x_i'\beta)^2$ subject to the constraint that (in
      the previous example) $\beta_2 + \beta_3 = 0$.
\begin{itemize}
\item set up lagrangian: $\sum_i (y_i - x_i'\beta)^2 + (\beta_2 + \beta_3)\lambda$
\item gives foc:
\begin{itemize}
\item $(0, 1, 1, 0) \cdot \beta = 0$
\begin{itemize}
\item rewrite this as $R\beta = 0$
\end{itemize}
\item $0 = - 2 \sum_i x_i (y_i - x_i'\beta) + (0, 1, 1, 0)' \lambda$
\begin{itemize}
\item can be written $0 = 2 X'X\beta - 2 X'Y + R' \lambda$
\end{itemize}
\end{itemize}
\item we can start to solve:
\begin{itemize}
\item $\hat\beta_R = \hat\beta - 1/2 (X'X)^{-1} R' \hat\lambda_R$
\item $R \hat\beta_R = 0$ (scalar)
\begin{itemize}
\item $= R \hat\beta - 1/2 R (X'X)^{-1} R' \hat\lambda_R$
\end{itemize}
\end{itemize}
\item so $- \hat\lambda_R/2 = (R (X'X)^{-1} R')^{-1} R\hat\beta$
\item $\hat\beta_R = \hat\beta - (X'X)^{-1} R' [R(X'X)^{-1}R']^{-1} R\hat\beta$
\item For intuition, look at the fitted values
\end{itemize}
\end{itemize}
\paragraph{diagram that should give intuition:}
\subsection{LM restrictions on a general example}
\label{sec-1-3}

\begin{itemize}
\item suppose now that we have the restriction $R\beta = q$
\item take first order conditions for the lagrangian
      \[ (Y - X\beta)'(Y - X\beta) + (R\beta - q)'\lambda \]
      (now \$$\lambda$ is a vector)
\begin{itemize}
\item $R\beta = q$
\item $0 = 2 X'X - 2 X'Y + R' \lambda$
\end{itemize}
\item mimic the same steps as before
\begin{itemize}
\item $\hat\beta_R = \hat\beta - {1\over2} (X'X)^{-1} R'\hat\lambda_R$
\item $-{1\over2} \hat\lambda_R = (R(X'X)^{-1}R')^{-1}(q - R\hat\beta)$
\item $\hat\beta_R = \hat\beta + (X'X)^{-1} R'(R(X'X)^{-1}R')^{-1}
        (q - R\hat\beta)$
\end{itemize}
\end{itemize}
\subsection{remarks}
\label{sec-1-4}

\begin{itemize}
\item this is also the \textbf{restricted MLE} under normality
\item estimator of $\sigma^2$ under this restriction is going to be ${1
      \over n - k - 1 + rank(R)} \sum_i (y_i - x_i'\hat\beta_R)^2$
\end{itemize}
\section{Hypothesis testing in finite samples}
\label{sec-2}

\begin{itemize}
\item Reading for this material: Greene 4.7, 5.1-5.3
\end{itemize}
\subsection{introduction}
\label{sec-2-1}

\begin{itemize}
\item Suppose that we want to test an arbitrary linear hypothesis about
       $\beta$; ie \[R \beta = q\] against the alternative \[R \beta \neq q\]
\begin{itemize}
\item ie $R = (1, 0, 0, \dots)$ and $q = 0$ gives us a test that
         $\beta_0 = 0$
\item $R = I$ and $q = (0,0,\dots,0)$ gives us a test that all of the
         coefficients are equal to zero.
\end{itemize}
\item for now, assume we have normal errors
\end{itemize}
\subsection{a convenient test}
\label{sec-2-2}
\paragraph{change in SSR under the null}
\label{sec-2-2-1}

      we can look at the change in the SSR when we impose the null
        hypothesis
\begin{itemize}
\item ie ${SSR_R - SSR \over SSR}$
\begin{itemize}
\item $SSR = \sum_i \hat\varepsilon_i^2$
\item don't need to present this, but can be written as \[
            {(R^2 - R^2_R) / J \over (1-R^2) / (n-k-1)} \]
\end{itemize}
\item Our test is actually a scaled version of that:
          \[ F = {(SSR_R - SSR)/J \over SSR / (n-k-1)} \]
\begin{itemize}
\item $J$ is the number of restrictions (ie dimension of $q$)
\end{itemize}
\end{itemize}
\paragraph{Distribution of F under null}
\label{sec-2-2-2}

      now, suppose the null is true, what is the distribution of $F$?
\paragraph{distribution of numerator}
\label{sec-2-2-2-1}

\begin{itemize}
\item reexpress the numerator
\begin{itemize}
\item $SSR_R - SSR = (Y - X\hat\beta_R)'(Y - X\hat\beta_R) - (Y -
             X\hat\beta)'(Y - X\hat\beta)$
\item $= (\hat\beta - \hat\beta_R)'X'X(\hat\beta - \hat\beta_R)$
             (you're proving this for homework)
\item remember that $\hat\beta_R = \hat\beta + (X'X)^{-1}
             R'(R(X'X)^{-1}R')(q - R\hat\beta)$
\item $= (q -
             R\hat\beta)'(R(X'X)^{-1}R')^{-1}R(X'X)^{-1}X'X(X'X)^{-1}R'(R(X'X)^{-1}R')^{-1}
             (q - R\hat\beta)$
\item $= (q -
             R\hat\beta)'(R(X'X)^{-1}R')^{-1}R(X'X)^{-1} R'(R(X'X)^{-1}R')^{-1}
             (q - R\hat\beta)$ after middle cancels
\item $= (q -
             R\hat\beta)'(R(X'X)^{-1}R')^{-1}(q - R\hat\beta)$
\end{itemize}
\item distribution of numerator
\begin{itemize}
\item $q - R\hat\beta$ is normal with mean $q - R\beta$ and
             variance $\sigma^2 R(X'X)^{-1}R'$
\item under the null, this mean is zero.
\item so we have a normal divided by its variance covariance matrix\ldots{}
\item so $(q - R\hat\beta)'(R(X'X)^{-1}R')^{-1}(q - R\hat\beta)$
             equals $\sigma^2$ chi-square r.v. with $J$ degrees of freedom
\end{itemize}
\end{itemize}
\paragraph{distribution of denominator}
\label{sec-2-2-2-2}

\begin{itemize}
\item just like earlier, we know that $SSR = (n-k-1) s^2$ and $s^2$
         and $\hat \beta$ are independent given $X$.
\item denominator is $\sigma^2$ times a chi-square with $n-k-1$
         degrees of freedom and indpendent of the numerator.
\end{itemize}
\paragraph{distribution of statistic}
\label{sec-2-2-2-3}

       distribution of $F$
\begin{itemize}
\item $F = {\sigma^2 \chi_J^2 / J \over \sigma^2 \chi_{n-k-1}^2 /
         (n-k-1)}$ in distribution.
\item numerator and denominator are independent
\item so this has an $F_{J, n-k-1}$ distribution under the null.
\end{itemize}
\paragraph{Distribution of F under alternative}
\label{sec-2-2-3}

\begin{itemize}
\item Denominator is not affected
\item Numerator is
\begin{itemize}
\item for $R\hat\beta - q$ to have mean zero, we need the null to
          be true
\item otherwise the numerator will get larger
\end{itemize}
\end{itemize}
\subsection{t-test}
\label{sec-2-3}

\begin{itemize}
\item Suppose you wanted to test a single restriction, say $\beta_i =
       b$ for some known $b$.
\item We know that ${\hat\beta_i - \beta_i \over \sqrt{s^2 q_i}}$ is
       the ratio of a standard normal r.v. and a chi-square/(n-k-1)
       random variable
\item so it is $t$ with (n-k-1) degrees of freedom
\item under the null, we know $\beta_i = b$, so we also have
       ${\hat\beta_i - b \over \sqrt{s^2 q_i}}$
\item we can use this as a test statistic:
\begin{itemize}
\item calculate the value of the r.v.
\item get the appropriate critical values from the t-distribution
         table (or the computer)
\item reject if the statistic is farther from zero than the
         critical value
\end{itemize}
\item If we're testing an equality, this will give us exactly the
       same test as the F-test with 1 and $n-k-1$ degrees of freedom
\item if we're testing an inequality, this test can be a little
       easier to work with.
\end{itemize}
\subsection{likelihood ratio test}
\label{sec-2-4}

\begin{itemize}
\item how would we test this hypothesis?
\begin{itemize}
\item maybe derive the likelihood ratio test.
\item remember, LRT is ${\sup_{\beta, \sigma^2: R\beta = q} L(\beta,
         \sigma^2; Y \mid X) \over \sup_{\beta, \sigma^2} L(\beta,
         \sigma^2; Y \mid X)}$ and we reject if this statistic is less
         than a critical value $c$.
\end{itemize}
\item we can plug in the MLEs
\begin{itemize}
\item $= {\hat\sigma^n \over \hat\sigma^n_{null}}
         exp((2\hat\sigma^2_a)^{-1} \sum_i (y_i -
         x_i'\hat\beta)^2 - (2\hat\sigma^2_0)^{-1} \sum_i (y_i - x_i'\hat\beta_0)^2)$
\item note that $\hat\sigma^2_a = n^{-1} \sum_i (y_i -
         x_i'\hat\beta_a)^2$
\item $\hat\sigma^2_0 = n^{-1} \sum_i (y_i - x_i'\hat\beta_0)^2$
\item so the terms in the exponent cancel and we're left with
         $(\hat\sigma^2 / \hat\sigma^2_R)^{n/2}$
\end{itemize}
\end{itemize}
\section{Hypothesis testing in the limit}
\label{sec-3}

\begin{itemize}
\item covered the material on \textit{2009-11-04 Wed} (I think)
\end{itemize}
\subsection{Testing linear hypotheses}
\label{sec-3-1}
\paragraph{t-test}
\label{sec-3-1-1}
\paragraph{justifying the t-test}
\label{sec-3-1-1-1}

\begin{itemize}
\item what does the previous asymptotic result tell us about the t-statistic?
\item suppose we want to test the null hypothesis
         \[ \beta_j \leq 1 \] so we construct the t-statistic
\item can we use it?
\item we know that, if the null is true,
         \[ {\beta_j - 1 \over \sqrt{s^2 q_i}} \to N(0,1) \] in
         distribution (where $q_i$ is the ii element of $(X'X)^{-1}$)
\item we also know that $t_{n-k-1} \to N(0,1)$ as $n\to\infty$
\begin{itemize}
\item \[t_{n-k-1} = {Z_0 \over \sqrt{(n-k-1)^{-1} \sum_{i=1}^{n-k-1}
           Z_i^2}}\] in distribution, where each of the $Z_i$ is an
           independent standard normal.
\item The sum in the denominator obeys a law of large numbers, so
           it converges to its mean, 1, in probability
\end{itemize}
\end{itemize}
\paragraph{interpretation/conclusion}
\label{sec-3-1-1-2}

\begin{itemize}
\item We have two approaches for testing the hypothesis that
         $\hat\beta_i = \beta_i^*$ (or has an inequality) based on
         asymptotic approximations
\begin{enumerate}
\item calculate ${\hat\beta_i - \beta_i \over \sqrt{s^2 q_i}}$
            and compare to critical values from the normal distribution.
\item calculate and compare to critical values from the
            $t$-distribution.
\end{enumerate}
\item These will \textbf{almost always} give you the same answer.
\begin{itemize}
\item the critical values for the t-distribution will always be a
           little bit bigger
\item this happens because we replace the scaled chi-squared
           distribution in the denominator with the constant 1
\end{itemize}
\item If they \textbf{don't} give you the same answer, watch out.
\begin{itemize}
\item you should probably go with the t-distribution (since it's
           more conservative)
\item this discrepency is telling you that you don't have enough
           observations to get reliable asymptotic approximations
\item doesn't mean that you should \textbf{believe} the errors are
           normal, but it means that you should probably play it safe
           and \textbf{not reject}
\item If you run into this situation, you'll want to think about
           what you're doing a little more carefully.
\end{itemize}
\end{itemize}
\paragraph{F test/Wald test}
\label{sec-3-1-2}
\paragraph{introduction}
\label{sec-3-1-2-1}

\begin{itemize}
\item Suppose we want to test the general restriction $R\beta = q$
\item If the errors are normal, we know that the F-test is valid
\item What is the asymptotic distribution of the F-test if the
         errors aren't normal?
\end{itemize}
\paragraph{asymptotic distribution}
\label{sec-3-1-2-2}

       We can rewrite the F-statistic as
       \[ J^{-1} (R\hat\beta - q)'[s^2 R(X'X)^{-1}R']^{-1}(R\hat\beta -
       q) \]
\begin{itemize}
\item under the null, we know that $q = R\beta$, so we also know
         that $R(\hat\beta - \beta) = R\hat\beta - q$
\item we know from earlier that $\sqrt{n} R(\hat\beta - \beta)$
         converges in distribution to $N(0, \sigma^2 R Q^{-1} R')$
\item so, we know that \[\sqrt{n} (R\hat\beta - q)' (\sigma^2 R
         Q^{-1} R')^{-1} \sqrt{n} (R\hat\beta - q) \] converges in
         distribution to a chi-square with $J$ degrees of freedom
\begin{itemize}
\item $\sqrt{n}(R\hat\beta - q)$ is a $J$-vector and is asymptotically
           normal.
\item $\sqrt{n}(\sigma^2 R Q^{-1} R')^{-1/2} (R\hat\beta - q)$ is
           an asymptotically standard normal $J$ vector.
\end{itemize}
\item we want to show that \[ (R\hat\beta - q)'[s^2
         R(X'X)^{-1}R']^{-1}(R\hat\beta - q) = J^{-1} (R\hat\beta -
         q)'[\sigma^2 RQ^{-1}R']^{-1}(R\hat\beta - q) + o_p(1)\]
\begin{itemize}
\item this will show that $J F$ is asymptotically chi-square with
           $J$ degrees of freedom.
\item multiply $(R\hat\beta - q)$ by $\sqrt{n}$ and bring $n^{-1}$
           next to $X'X$.
\item we know that $s^2 \to \sigma^2$ and $n^{-1} X'X \to Q$ in probability
\item so, we get the convergence we need.
\end{itemize}
\item So, we know that the F statistic is equal in distribution
         to a $J^{-1} \chi_J^2$ random variable as $n \to \infty$.
\end{itemize}
\paragraph{conclusion/interpretation}
\label{sec-3-1-2-3}
\paragraph{behavior of F distribution}
\label{sec-3-1-2-3-1}

\begin{itemize}
\item have the same issue we saw with the t-distribution
\item the $F_{J, n-k-1}$ distribution is defined to be the same as
          the distribution of
          \[{J^{-1} \sum_{i=1}^J Z_i^2 \over (n-k-1)^{-1} \sum_j
          W_j^2}\] where each $Z$ and $W$ is an independent standard normal.
\item as $n\to \infty$, the denominator converges to 1 in probability
\item the numerator is unaffected and is a $\chi^2_J / J$ random variable.
\item so the F-distribution becomes more and more like the chi
          square as well.
\end{itemize}
\paragraph{conclusion}
\label{sec-3-1-2-3-2}

\begin{itemize}
\item You can use either the $F_{J, n-k-1}$ or the Chi-square $J$ to
          construct a hypothesis test.
\begin{itemize}
\item calculate the F test statistic
\item get the critical values from a table or the computer
\item reject if test stat is greater than critical values
\end{itemize}
\item Just like with t-test, using the finite sample test (F) is
          more conservative than the asymptotic test
\begin{itemize}
\item critical values are a little bigger
\item this is a bigger deal when you're testing lots of
            restrictions simultaneously, so you are more likely to get
            disagreement than you did with the t-test
\item \textbf{use the F critical values}
\end{itemize}
\end{itemize}
\subsection{Testing nonlinear hypotheses}
\label{sec-3-2}

     Reading is based on Greene's section 5.5.
\paragraph{motivation}
\label{sec-3-2-1}

      What if we want to test the hypothesis that $c(\beta) = q$ where
      $c$ is a function from $\mathbb{R}^{k+1}$ to $\mathbb{R}^J$?
\begin{itemize}
\item Do we know anything about the finite-sample distribution of
        $c(\hat\beta)$ that we could use to construct a test statistic?
\begin{itemize}
\item as usual, if we specify a distribution for the errors, we
          could crank through the algebra and derive the distribution of
          $c(\hat\beta)$
\item won't give us a general procedure
\item won't necessarily be easy to impose the null hypothesis.
\end{itemize}
\item Maybe we can get a procedure that uses the asymptotic
        distribution of $c(\hat\beta)$ under the null
\end{itemize}
\paragraph{initial asymptotics}
\label{sec-3-2-2}

\begin{itemize}
\item suppose that $c$ is differentiable
\item we can use the delta-method to show that
        $\sqrt{n}(c(\hat\beta) - c(\beta))$ is asymptotically normal
\item a first-order taylor expansion gives us
        \[ \sqrt{n} c(\hat\beta) = \sqrt{n} c(\beta) +
        {\partial c(\beta) \over \partial \beta'} \sqrt{n} (\hat\beta -
        \beta) + o_p(1) \]
        where $\partial c(\beta) \over \partial \beta'$ is a $J \times
        k+1$ matrix of partial derivatives.
\item We know that \[{\partial c(\beta) \over \partial \beta'}
        \sqrt{n} (\hat\beta - \beta)\] converges in distribution to a
        normal r.v. with mean zero and variance \[ \left({\partial
        c(\beta) \over \partial \beta'}\right) \sigma^2 Q^{-1}
        \left({\partial c(\beta) \over \partial \beta'}\right)'\]
\item Then $\sqrt{n} (c(\hat\beta) - c(\beta))$ also converges to \[
        N\left(0, \left({\partial
        c(\beta) \over \partial \beta'}\right) \sigma^2 Q^{-1}
        \left({\partial c(\beta) \over \partial \beta'}\right) \right)\]
\end{itemize}
\paragraph{hypothesis testing}
\label{sec-3-2-3}

\begin{itemize}
\item Under the null hypothesis, we know that $c(\beta) = q$, so
        $\sqrt{n}(c(\hat\beta) - q)$ has the same asymptotic
        distribution as $\sqrt{n}(c(\hat\beta) - c(\beta))$
\item so we can do the same trick as before to get a chi-square test
        statistic
\begin{itemize}
\item we estimate the variance with \[ \left({\partial
        c(\hat\beta) \over \partial \hat\beta'}\right)s^2 (n^{-1}X'X)^{-1}
        \left({\partial c(\hat\beta) \over \partial \hat\beta'}\right)\]
\begin{itemize}
\item know $\hat\beta \to \beta$ so $\left({\partial c(\hat\beta)
            \over \partial \hat\beta'}\right) \to \left({\partial
            c(\beta) \over \partial \beta'}\right)$ in probability
\item know that $s^2 (n^{-1} X'X)^{-1} \to \sigma^2Q^{-1}$ in
            probability
\end{itemize}
\[{c(\hat\beta) \over \partial \hat\beta'}\Big)s^2 (n^{-1}X'X)^{-1}
        \left({\partial c(\hat\beta) \over \partial \hat\beta'}\right)\]
          - know $\hat\beta \to \beta$ so $\left({\partial c(\hat\beta)
            \over \partial \hat\beta'}\right) \to \left({\partial
            c(\beta) \over \partial \beta'}\right)$ in probability
          - know that $s^2 (n^{-1} X'X)^{-1} \to \sigma^2Q^{-1}$ in
            probability
\end{itemize}
\[{c(\hat\beta) \over \partial \hat\beta'}\Big)s^2 (n^{-1}X'X)^{-1}
        \left({\partial c(\hat\beta) \over \partial \hat\beta'}\right)\]
          - know $\hat\beta \to \beta$ so $\left({\partial c(\hat\beta)
            \over \partial \hat\beta'}\right) \to \left({\partial
            c(\beta) \over \partial \beta'}\right)$ in probability
          - know that $s^2 (n^{-1} X'X)^{-1} \to \sigma^2Q^{-1}$ in
            probability
\item Our test statistic is 
        \[ (c(\hat\beta) - q)' \left[\left({\partial
        c(\hat\beta) \over \partial \hat\beta'}\right)s^2 (n^{-1}X'X)^{-1}
        \left({\partial c(\hat\beta) \over \partial \hat\beta'}\right) \right]^{-1}
        (c(\hat\beta) - q) \]
\item asymptotically chi-square with $J$ degrees of freedom
\item one additional point
\begin{itemize}
\item we need to verify that the population variance matrix is
          invertible
\item ie we need \[\left({\partial c(\beta) \over \partial
          \beta'}\right)\] to have full rank
\item in ``standard'' applications this is true, but it \textbf{may} not be true
\item can use a similar procedure but higher order expansion and
          usually won't wind up with a chi-square statistic.
\end{itemize}
\end{itemize}
\section{Testing multiple hypotheses}
\label{sec-4}

\begin{itemize}
\item Fill in
\end{itemize}
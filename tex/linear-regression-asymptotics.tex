\documentclass[11pt]{article}
\usepackage[utf8]{inputenc}
\usepackage[T1]{fontenc}
\usepackage{amssymb}
\usepackage{amsmath}
\usepackage{fixltx2e}
\usepackage{graphicx}
\usepackage{longtable}
\usepackage{float}
\usepackage{wrapfig}
\usepackage{soul}
\usepackage{textcomp}
\usepackage{marvosym}
\usepackage{wasysym}
\usepackage{latexsym}
\usepackage{hyperref}
\tolerance=1000
\usepackage[margin=1in]{geometry}
\setcounter{secnumdepth}{8}
\providecommand{\alert}[1]{\textbf{#1}}

\title{Asymptotic properties of linear regression}
\author{Gray Calhoun}
\date{\today}
\hypersetup{
  pdfkeywords={},
  pdfsubject={},
  pdfcreator={Emacs Org-mode version 7.8.02}}

\begin{document}

\maketitle

\setcounter{tocdepth}{2}
\tableofcontents
\vspace*{1cm}
\begin{itemize}
\item Reading for this material: Greene 4.9, 5.4, 5.5
\item cover on
\begin{itemize}
\item \textit{2009-10-28 Wed}, \textit{2009-11-02 Mon}
\item \textit{2010-11-02 Tue}, \textit{2010-10-28 Thu},
\item \textit{2011-10-27 Thu}
\item \textit{2012-10-25 Thu}
\end{itemize}
\end{itemize}
\section{intro}
\label{sec-1}

\begin{itemize}
\item we've gone back and forth on assuming normality
\item some of the properties of the OLS estimator require normality
\item others don't
\end{itemize}
\subsection{OLS properties without normality}
\label{sec-1-1}

\begin{itemize}
\item algebraic properties from chapter 3
\item unbiasedness of the estimators
\item formula for variance of the estimators
\item gauss-markov theorem
\item unbiasedness of $s^2$
\item zero correlation between errors and beta-hat
\end{itemize}
\subsection{Properties of OLS that use normality}
\label{sec-1-2}

\begin{itemize}
\item distribution
\item independence of errors and beta-hat
\item hypothesis testing and confidence intervals
\item does this mean we can't do hypothesis testing if we don't have
       normality?
\begin{itemize}
\item of course not
\end{itemize}
\end{itemize}
\subsection{asymptotics}
\label{sec-1-3}

\begin{itemize}
\item we could plug in any distribution for $\epsilon$ and then use
       transformation formulas to get the density of $\hat\beta$ given $X$.
\item each distribution for the errors would give us a different
       distribution for $\hat\beta$.
\item \textbf{open up R}
\begin{itemize}
\item make a function 
         getbeta = function(errdist, nerrs, nbetas = 800) \{
         errs = matrix(errdist(nerrs * nbetas, nerrs, nbetas))
         apply(errs, 2, mean)
         \}
\item plot hist for different densities with nerrs = 50
\begin{itemize}
\item rt (errdist = function(n) rt(n, df = 10))
\item rexp
\item runif
\end{itemize}
\end{itemize}
\item notice that these are pretty close to each other
\item so we could do one of two things in practice
\begin{itemize}
\item try to guess what the distribution of the errors is and
         calculate the distribution of beta-hat from that
\item try to choose a distribution for beta-hat that's going to be
         close to \textbf{most} of the different dists you'll get for
         different error dist
\end{itemize}
\item big insight
\begin{itemize}
\item these procedures are the same
\item it turns out that the distribution of beta-hat \textbf{usually}
         behaves pretty similar to the one you get assuming normal
         errors, even if the errors are not normal
\end{itemize}
\item a side point, you'll never get approximate distributions
       that are good for all error dists.
\begin{itemize}
\item do example with rcauchy
\end{itemize}
\end{itemize}
\section{asymptotic behavior of beta-hat}
\label{sec-2}

\begin{itemize}
\item assumptions
\begin{itemize}
\item $Y = X\beta + \epsilon$
\item $\epsilon$ is iid $(0, \sigma^2)$ given $X$ and has finite
       $2+\delta$ moments for some positive number $\delta$
\item $x_i$ is independent across $i$
\item for each i, $x_i x_i'$ has uniformly finite variance
\item $E x_i x_i' = Q_i$ which is uniformly finite and positive definite.
\end{itemize}
\item conclusion
\begin{itemize}
\item $\hat\beta \to \beta$ in probability
\item $s^2 \to \sigma^2$ in probability
\item $\sqrt{n}(\hat\beta - \beta) \to N(0, \sigma^2 Q^{-1})$ in
       distribution.
\end{itemize}
\end{itemize}
\subsection{discussion of assumption on X'X}
\label{sec-2-1}

\begin{itemize}
\item Greene makes the assumption that $n^{-1} X'X \to Q$ in
       probability where $Q$ is a positive definite matrix.
\item this condition is implied by the one we're using
\begin{itemize}
\item $n^{-1} X'X = n^{-1} \sum_i x_i x_i'$, so define $Q =
         \lim_{n\to\infty} n^{-1} \sum_i Q_i$ and we get $n^{-1} \sum_i
         x_i x_i' - Q \to 0$ in probability as a consequence of
         Chebychev's weak law of large numbers
\end{itemize}
\end{itemize}
\subsection{proof of consistency}
\label{sec-2-2}

     We want to prove that $\hat\beta \to \beta$ in probablilty.  We
     can do that by plugging through the algebra
\begin{itemize}
\item going to show that $\hat\beta - \beta \to 0$ in probability
\item $\hat\beta - \beta = (\sum_i x_i x_i')^{-1} \sum_i x_i
       \varepsilon_i$
\begin{itemize}
\item $= (n^{-1} \sum_i x_i x_i')^{-1} n^{-1} \sum_i x_i \epsilon_i$
\item $= \left[(n^{-1} \sum_i x_i x_i')^{-1} n^{-1}  -
         Q^{-1}\right] \sum_i x_i \epsilon_i + Q^{-1} n^{-1} \sum_i
         x_i \epsilon_i$
\item $= O_p(1) n^{-1} \sum_i x_i \epsilon_i$
\begin{itemize}
\item remember what $n^{-1} \sum_i x_i x_i' = O_p(1)$ means -- it
           means that for each element of the matrix (we'll call the ij
           element $(n^{-1} X'X)_{ij}$) for every positive \$$\delta$,
           there exists a finite constant $C_\delta$ and a number
           $N_\delta$ such that \[ P[|(n^{-1} X'X)_{ij} | > C_\delta] <
           \delta \] for all $n > N_\delta$
\item we know from chebychev's inequality that \[ P[|(n^{-1}
            X'X)_{ij} | > C] < E |(n^{-1} X'X)_{ij} |^2 / C^2\]
\item $\leq n^{-1} \sum_t E |(x_{it} x_{jt}| / C^2$
\item this is finite because we are assuming finite means and variances.
\item so, for any $\delta$, we just choose $N$ and $C$ big enough
           to ensure that the inequality holds for all $n > N$.
\end{itemize}
\item now, $E(x_i \epsilon_i) = E E(\epsilon_i \mid x_i) = 0$
\item also, $var(x_i \epsilon_i) = E(x_i x_i' \epsilon_i^2)$
\begin{itemize}
\item $= E E(x_i x_i' \epsilon_i^2 \mid x_i)$
\item $= \sigma^2 E(x_i x_i')$
\item $= \sigma^2 Q_i$
\end{itemize}
\item so $n^{-1} \sum_i x_i \epsilon_i$ satisfies Chebychev's LLN
         and converges to its mean, zero.
\end{itemize}
\item $\hat\beta - \beta = O_p(1) o_p(1) = o_p(1)$ and converges to
       zero in probability.
\end{itemize}
\subsection{consistency of $s^2$ \textbf{:hw:}}
\label{sec-2-3}

     we want to show that $s^2 \to \sigma^2$ in probability
\begin{itemize}
\item $s^2 = {1 \over n - k - 1} \sum_i (y_i - x_i'\hat\beta)^2$
\begin{itemize}
\item $= {1 \over n - k - 1} \sum_i (\epsilon_i - x_i'(\hat\beta-\beta))^2$
\item $= {1 \over n - k - 1} \sum_i (\epsilon_i^2 - 2 \epsilon_i
         x_i'(\hat\beta-\beta) + (x_i'(\hat\beta-\beta))^2)$
\item \[= {1 \over n - k - 1} \sum_i \epsilon_i^2 - 2 (\hat\beta -
         \beta)' {1 \over n - k - 1} \sum_i  x_i \epsilon_i +
         (\hat\beta - \beta)' {1 \over n - k - 1} \sum_i x_i x_i'
         (\hat\beta - \beta)\]
\item $=  {n \over n - k - 1} n^{-1} \sum_i \epsilon_i^2 + o_p(1)
         o_p(1) + o_p(1)'O_p(1)o_p(1)$
\item $= n^{-1} \sum_i \epsilon_i^2 + o_p(1)$
\end{itemize}
\item finally, $n^{-1} \sum_i \epsilon_i^2$ converges in probability
       to $\sigma^2$ by Markov's lln.
\end{itemize}
\subsection{asymptotic normality of $\hat\beta$}
\label{sec-2-4}

     We'll do a similar argument for asymptotic normality
\begin{itemize}
\item $\sqrt{n}(\hat\beta - \beta) = (n^{-1} \sum_i x_i x_i')^{-1}
       n^{-1/2} \sum_i x_i \epsilon_i$
\begin{itemize}
\item $= ((n^{-1} \sum_i x_i x_i')^{-1} - Q^{-1}) n^{-1/2} \sum_i
         x_i \epsilon_i + Q^{-1} n^{-1/2} \sum_i x_i \epsilon_i$
\end{itemize}
\item so, we're going to show two things
\begin{itemize}
\item $n^{-1/2} \sum_i x_i \epsilon_i$ is asymptotically normal
\item $((n^{-1} \sum_i x_i x_i')^{-1} - Q^{-1}) \to 0$ in
         probability (which we discussed earlier)
\end{itemize}
\item for normality
\begin{itemize}
\item we already showed that each $x_i \epsilon_i$ has mean zero and
         variance $\sigma^2 Q_i$
\item we can appy the lindberg-feller CLT to show $n^{-1/2} \sum_i
         x_i \epsilon_i$ converges in distribution to $N(0, Q)$
\end{itemize}
\item consequently, $\sqrt{n}(\hat\beta - \beta) = o_p(1) + Q^{-1}
       \sum_i x_i \epsilon_i$ and converges in distribution to a normal
       with mean zero and variance $\sigma^2 Q^{-1}$ which is just what
       we wanted to show.
\end{itemize}

\end{document}
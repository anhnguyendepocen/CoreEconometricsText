% Copyright © 2013, authors of the "Econometrics Core" textbook; a
% complete list of authors is available in the file AUTHORS.tex.

% Permission is granted to copy, distribute and/or modify this
% document under the terms of the GNU Free Documentation License,
% Version 1.3 or any later version published by the Free Software
% Foundation; with no Invariant Sections, no Front-Cover Texts, and no
% Back-Cover Texts.  A copy of the license is included in the file
% LICENSE.tex and is also available online at
% <http://www.gnu.org/copyleft/fdl.html>.

\input{VERSION}
\author{Gray Calhoun} 
% (This comment is repeated in the Makefile)
% I'm still not sure the best way to do author information; I'm much
% more concerned in the long run about how different attributation
% styles would make someone more or less likely to contribute to an
% existing text or to license an existing draft.  For now, there's
% only one author, so I'll put myself as the author.  If someone else
% contributes any edits, etc., I'll change it to {Gray Calhoun and
% EFLP}.  If anyone wants to contribute a lot of original material and
% wants named authorship, please email the mailing list so we can
% discuss merging projects.

\usepackage{amssymb,amsmath,verbatim}
\usepackage{fontspec,unicode-math,xltxtra,xunicode,booktabs}
\setromanfont[Ligatures=TeX]{TeX Gyre Pagella}
\setsansfont[Ligatures=TeX,Scale=MatchLowercase]{TeX Gyre Heros}
% \setmonofont[Scale=MatchLowercase]{Inconsolata}
\setmathfont{Asana-Math}

\frenchspacing
\setcounter{secnumdepth}{1}
\setcounter{tocdepth}{1}
\renewcommand\bibname{}
\renewcommand\refname{}
\renewcommand\contentsname{}
\bibliographystyle{abbrvnat}
\setcitestyle{round}
\newcommand{\email}[1]{\href{mailto:#1}{\nolinkurl{<#1>}}}

% Workaround for bug in the tufte-latex class
\renewcommand\smallcapsspacing[1]{{\addfontfeature{LetterSpace = 8}\scshape#1}}
\renewcommand\allcapsspacing[1]{{\addfontfeature{LetterSpace = 15}#1}}

% Math shortcuts
\renewcommand{\Pr}{\operatorname{Pr}}

\DeclareMathOperator{\1}{1}
\DeclareMathOperator{\abs}{abs}
\DeclareMathOperator{\avar}{avar}
\DeclareMathOperator{\bias}{bias}
\DeclareMathOperator{\corr}{corr}
\DeclareMathOperator{\cov}{cov}
\DeclareMathOperator{\E}{E}
\DeclareMathOperator{\median}{median}
\DeclareMathOperator{\mse}{mse}
\DeclareMathOperator{\rank}{rank}
\DeclareMathOperator{\range}{range}
\DeclareMathOperator{\sd}{sd}
\DeclareMathOperator{\tr}{tr}
\DeclareMathOperator{\var}{var}

\DeclareMathOperator*{\argmax}{arg\,max}
\DeclareMathOperator*{\argmin}{arg\,min}
\DeclareMathOperator*{\plim}{plim}

\DeclareMathOperator{\binomial}{binomial}
\DeclareMathOperator{\invWishart}{inverse\ Wishart}
\DeclareMathOperator{\N}{N}
\DeclareMathOperator{\uniform}{uniform}

\newcommand{\BB}{\ensuremath{\mathbb{B}}}
\newcommand{\NN}{\ensuremath{\mathbb{N}}}
\newcommand{\PP}{\ensuremath{\mathbb{P}}}
\newcommand{\QQ}{\ensuremath{\mathbb{Q}}}
\newcommand{\RR}{\ensuremath{\mathbb{R}}}
\newcommand{\RRᵏ}{\ensuremath{\mathbb{R}ᵏ}}
\newcommand{\RRⁿ}{\ensuremath{\mathbb{R}ⁿ}}
\newcommand{\RRb}{\ensuremath{\bar{\mathbb{R}}}}
\newcommand{\ZZ}{\ensuremath{\mathbb{Z}}}

\newcommand{\Fs}{\ensuremath{\mathcal{F}}}
\newcommand{\Gs}{\ensuremath{\mathcal{G}}}
\newcommand{\Ps}{\ensuremath{\mathcal{P}}}

\newcommand{\ov}[2][1]{\tfrac{#1}{#2}}
\newcommand{\iid}{i.i.d.}

\newcommand{\ep}{\varepsilon}
\newcommand{\eph}{\hat{\varepsilon}}

\newcommand{\ah}{\hat{a}}
\newcommand{\αh}{\hat{α}}
\newcommand{\bh}{\hat{b}}
\newcommand{\βb}{\bar{β}}
\newcommand{\βh}{\hat{β}}
\newcommand{\βt}{\tilde{β}}
\newcommand{\eh}{\hat{e}}
\newcommand{\εb}{\bar{ε}}
\newcommand{\εh}{\hat{ε}}
\newcommand{\εt}{\tilde{ε}}
\newcommand{\ηh}{\hat{η}}
\newcommand{\Fh}{\hat{F}}
\newcommand{\λh}{\hat{λ}}
\newcommand{\μb}{\bar{μ}}
\newcommand{\μh}{\hat{μ}}
\newcommand{\Ωh}{\hat{Ω}}
\newcommand{\Σh}{\hat{Σ}}
\newcommand{\sh}{\hat{s}}
\newcommand{\σh}{\hat{σ}}
\newcommand{\θh}{\hat{θ}}
\newcommand{\Vh}{\hat{V}}
\newcommand{\Xb}{\bar{X}}
\newcommand{\Xc}{\mathcal{X}}
\newcommand{\Xh}{\hat{X}}
\newcommand{\Xt}{\tilde{X}}
\newcommand{\Yb}{\bar{Y}}
\newcommand{\Yh}{\hat{Y}}
\newcommand{\Yt}{\tilde{Y}}
\newcommand{\yh}{\hat{y}}

\newcommand{\dx}{\,dx}
\newcommand{\dy}{\,dy}
\newcommand{\dμ}{\,dμ}
\newcommand{\dθ}{\,dθ}
\renewcommand{\dz}{\,dz}

\newtheorem{thm}{Theorem}[section]
\newtheorem{defn}{Definition}[section]
\newtheorem{ex}{Example}[section]
